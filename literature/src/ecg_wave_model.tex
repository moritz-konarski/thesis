\documentclass[../00_main.tex]{subfiles}

\begin{document}

\section{\textcite{f}}

\fullcite{f}

\subsubsection*{Summary}

\begin{itemize}
    \item propose a mathematical model of the ECG wave
    \item human body == cylindrical composite dielectric and conducting medium
    \item heart == harmonic bio--signal generator
    \item can use this model to predict experimental data
    \item ECG signal is due to heart beat and that is due to the signal of the
        S.A. node
    \item the electric field generated by this is then propagated to the
        surface through the dielectric medium that is the human body
    \item many different approaches to ECGs mentioned here, list here
    \item compression techniques to help with large amounts of data
    \item QRS complex evaluation in one paper
    \item the signal measured by an ECG electrode can be represented by Fourier
        harmonic components
    \item dense mathematical description of the model
    \item they test their model using an actual ECG -- get the Fourier
        components from it
    \item all 12 ECG leads have about the same makeup of Fourier components
    \item they have to randomly assign some of the values to make the model fit
    \item this model is more accurate because it assumes a cylindrical body and
        not a sphere like the classical models
    \item good source list
    \item IFR: do rigorous experimentation to really test this model;
\end{itemize}

\end{document}
