\documentclass[../00_main.tex]{subfiles}

\begin{document}

\section{\textcite{c}}

\fullcite{c}

\subsubsection*{Summary}

\begin{itemize}
    \item digital signal filtering methods for ECGs
    \item remove 50Hz network and breathing muscle artifacts
    \item 3 heart rate detection algorithms
    \item main problems with ECGs are interfering 50Hz supply signals and
        muscle artifacts
    \item for real--time applications, these things should be very efficient
    \item heart rate is important and can be computed from ECGs among other
        things
    \item often, heart rate is detected by measuring the distance between QRS
        complexes
    \item neural networks, genetic algorithms, wavelet transforms, filter
        banks, adaptive threshold, signal spectral analysis, short--term
        autocorrelation can be used to find it
    \item the methods here are simpler and real--time suited
    \item Butterworth filter is used in professional ECG applications
    \item they remove all the noise from the signal first, using the described
        methods
    \item Butterworth filters are used to also detect the R peaks
    \item with the highlighted R peaks one can detect heart rate
    \item for heart rate detection, the autocorrelation method can be use
        because R peaks are quasi--periodical
    \item other methods find the difference between R peaks, by either using
        thresholds, or peak detector
    \item the three algorithms find completely different results
\end{itemize}

\end{document}
