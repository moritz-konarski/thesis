\documentclass[../00_main.tex]{subfiles}

\begin{document}

\section{\textcite{e}}

\fullcite{e}

\subsubsection*{Summary}

Well written paper

\begin{itemize}
    \item signal processing and data analysis are widely used methods
    \item detecting cardiovascular abnormalities with an ECG is possible
    \item a fuzzy-based multi-objective algorithm using a fast fourier
        transform is used to extract rough features like PQRST amplitude
    \item then apply an algorithm to classify the abnormality
    \item ECG behavior depends on many different factors
    \item accuracy is achieved by taking into account these factors
    \item maintaining a database of previous results makes prediction better
    \item this provides 98.7\% efficiency in abnormality detection
    \item accurate ECGs are necessary to classify cardiac abnormalities
    \item ECGs are noisy and thus an algorithm needs to de-noise the signal
    \item after noise removal, ECG signals must be extracted -- FFT
    \item fuzzy--based scheme should classify how sick a patient is
    \item de-noising can be done using a wavelet transform
    \item contour wavelet transform CTW -- Daubechies algorithm for de-nosing
    \item goal is to remove all noise
    \item discrete wavelet transform is not accurate enough, adaptive wavelet
        decomposition is proposed
    \item then FFT is used to extract the features
    \item ANN for classification
    \item FFT to discretize the signal
    \item radix--2 FFT, is the simplest way to evaluate the DFT
    \item heartbeat is calculated as the interval between two R peaks --
        heartbeat is the number of R peaks in a particular minute
    \item RR interval can be useful for fining symptoms that include
        heart--rate variation
    \item QRS is the main thing that a heart's conditions is measured by
    \item QRS duration is the time interval between the two peak Q and
        S signals
    \item multi-objective genetic algorithm is exactly what it sounds like
    \item uses MIT--BIH arrhythmia database
    \item finds good results for their approach
    \item methods is more efficient than previous results
    \item IFR: analysing and modelling the sequence of heartbeats using 
        advanced machine learning methods can be implemented to achieve better 
        performance
\end{itemize}

\end{document}
