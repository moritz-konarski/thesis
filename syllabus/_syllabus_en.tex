\documentclass[a4paper, 12pt]{article}

\usepackage{pdfpages}
\usepackage[margin=1in]{geometry}
\usepackage{hyperref}
\hypersetup{
    colorlinks=true,
    linkcolor=blue,
    filecolor=magenta,
    urlcolor=cyan,
}
\urlstyle{same}

\usepackage{fontspec}
    \defaultfontfeatures{Renderer=Basic,Ligatures={TeX}}
    \setmainfont{CMU Serif}
    \setsansfont{CMU Sans Serif}
    \setmonofont{CMU Typewriter Text}
\usepackage[english,russian]{babel}

\begin{document}

\section*{Syllabus: Senior Thesis Seminar MAT-481\\Fall 2020, Spring 2021}

\begin{enumerate}
    \item \textbf{Instructor}: Sklyar Sergey Nikolaevich - Professor, Doctor of 
        Science in Physics and Mathematics, Office: 415, Phone: +998 (312) 
        91-50-00 (Ext: 426), E-mail: \url{sklyar_s@auca.kg}
    \item \textbf{Volume of academic load}: 1 meeting per week (75 minutes, 15 
        working weeks; load for students -- 3 credit hours in each semester)
    \item \textbf{The purpose of the seminar} is to help students choose the 
        direction of their research, topics of senior theses, and thesis 
        supervisors; you will also get support and consultations in the process 
        of preparation and design of the senior thesis
    \item \textbf{Assessment}: the grade is determined by attendance of the 
        seminar and the work for the thesis preparation.
    \item \textbf{Highlights of the seminar}:
        \begin{itemize}
            \item \textit{September-October}: Presentations of the AMI 
                Department professors about their research. As part of these 
                presentations, teachers also formulate topics that are offered 
                to students for research and prospective senior theses; 3rd 
                year students of the AMI Program are also invited to these 
                sessions.
            \item \textit{November-December}: Reports and presentations of the 
                4th year AMI students as the result of Internship (educational 
                and work experience).
            \item \textit{February-April}: Consultations on senior theses, 
                finalization of the results, preparation of presentations for 
                thesis defense at the State Attestation Committee.
        \end{itemize}
\end{enumerate}

\newpage

\section*{Structure and Rules of the Thesis}

AUCA Bachelor's final qualifying work is written in the English language. The 
structure of the work includes a number of mandatory and optional elements. 
Mandatory elements of the work are as follows:

\begin{itemize}
    \item title page
    \item abstract
    \item table of contents
    \item introduction
    \item main part of the document
    \item conclusion
    \item reference list
    \item additional elements
        \begin{itemize}
            \item acknowledgements
            \item glossary (conventions, abbreviations, terms)
            \item appendix
        \end{itemize}
\end{itemize}

\section*{Description of the Structural Elements}

A sample \textbf{cover page} is provided, it is not numbered.\\

An \textbf{abstract} is a short description of the work performed. The purpose of 
the abstract is to give the reader the first idea of the essence and the main 
results of the work. The abstract is \textbf{located on the second sheet} of work, 
its volume \textbf{does not exceed one page, generally half a page}. Examples of 
abstracts are presented on page 7.\\

The \textbf{table of contents} includes a listing of \textbf{all sections and
subsections}  of the work, starting immediately after the table of contents. 
Table of contents lines end with the numbers of pages where the beginning of 
the corresponding part of the document is located. Examples of the table of 
contents is presented on pages 8 and 9.\\

The \textbf{introduction} contains a short verbal statement of the problem and 
provides a rational relevance of the selected topic. The introduction 
also \textbf{formulates goals and objectives}, implemented in the process of working
on the project. Already in the introduction, you can refer to information 
sources, if necessary to substantiate the relevance of the topic work. 
Sometimes the introduction provides a brief description of main results.\\

The \textbf{main part} of the thesis contains an overview of the current state of 
the problem, based on on the analysis of information (literary) sources, 
formulation of the solved (investigated) problems, a review of methods for 
solving the problem under consideration and related problems, as well as 
description of the research results carried out personally by the author of the
thesis, including experiments, if they are required on the assignment for the 
thesis. The main part of the thesis is usually divided into sections and 
subsections (chapters and paragraphs), which are numbered in Arabic numerals: 
“2. Method ... " - the second section (chapter); "2.1 Shear Stress ..." - 
the first subsection (paragraph) of the second section; "2.1.2 Wall Shear 
Stress ..." - the second part of the first subsection, the second section, 
etc. The titles of sections and subsections should be worded concisely and
reflect their content.\\

The \textbf{first section} (chapter) of the main part of the thesis contains an 
overview of the current state of problems based on the analysis of 
information (literary) sources. When linking to a source of information in 
this section should not attempt to characterize the cited article, monograph, 
study guide, or information of the Internet resource in in general: it should 
be noted those ideas and results that are directly related to the topic of your
work. References to literary sources can be continued as well can be repeated,
as necessary, in other sections of the main part of the work. It is not 
recommended to call this section "Literary Review", it is better to come up 
with a name, reflecting the theme of the thesis, for example, “Mathematical 
models and methods of magnetotelluric monitoring". Otherwise, the content of 
the main part of the thesis is completely determined by the author.\\

In the \textbf{conclusion}, the work done is summarized: the main results are 
formulated, their scientific novelty and practical significance (if any) are 
noted. Maybe also indicate the prospects for the development of the considered
direction. Examples of conclusion can be found on pages 10 and 11.\\

List of information sources (\textbf{list of references}) contains the bibliographic
description of all information sources used in the process of working on the 
thesis. Don't list articles or monographs on which there is not a single 
reference in the text of the work. Use a variety of forms of bibliographic 
description, including and specially structured options. We offer a simplified
structure: \textbf{information about each of the sources in the List is arranged in 
the order of their mention in the text} and \textbf{formulated in the original 
languages}. When mentioned in the text, literary source numbered with an
Arabic numeral in square brackets (for example, [13]), under the same 
number this the source is also on the List. Some examples are given on pages 
5 and 6.\\

If the above additional elements are present in the thesis structure, then
Acknowledgments and Glossary appear immediately after the table of contents 
in the order shown. Appendix is located at the very end, after references. The 
Appendix contains materials which were not included in the main text of the 
work, but which allow to reveal more fully the essence of conducted research. 
It can be additional tables, figures, diagrams complex algorithms, program 
printouts, proofs of auxiliary statements, etc.\\

\section*{Rules for the Thesis}

\subsection*{General Rules}

\begin{itemize}
    \item for the design of the thesis, either MS Word with MathType, or
        \LaTeX{} are recommended
    \item the minimum length of the thesis (including all annexes) should be 25 
        pages, the maximum length 60 pages
    \item the thesis is printed on A4 paper, vertical orientation, single-sided
        printing only
    \item fonts must be consistent, except for tables and formula indices, they 
        may be smaller. Section titles may be typed in capital letters, italics 
        or boldface may be used for emphasis
    \item new section must begin on a new sheet
    \item sheet numbers are placed on the outer bottom corner, the title page 
        and pages completely occupied by figures can not be numbered, but must 
        be counted
    \item figures should be numbered, captioned, and referenced in the text. 
        Numbering can be continuous or section internal
    \item formulas must only be numbered if they are referenced in the text
\end{itemize}

\subsection*{Margins}

\begin{center}
    \begin{tabular}{| c | c |}
        \hline
        Page Side & Margin, cm \\
        \hline
        Left & 3 \\
        \hline
        Upper & 2 \\
        \hline
        Lower & 2 \\
        \hline
        Right & 1.5 \\
        \hline
    \end{tabular}
\end{center}

\subsection*{Typesetting}

\begin{center}
    \begin{tabular}{| c | c |}
        \hline
        Parameter & Value \\
        \hline
        Font Size & 12 pt  \\
        \hline
        Font & Times New Roman \\
        \hline
        Indentation & 1.25cm \\
        \hline
        Line Spacing & 1.5 \\
        \hline
        Text Alignment & Block Text \\
        \hline
    \end{tabular}
\end{center}

\subsection*{Bibliography Formatting Examples}

\subsubsection*{Books}

\begin{enumerate}
    \item Stommel H. The Gulf Stream. –Univ.California Press, 1965. -243 p.
    \item Марчук Г.И., Саркисян А.С. Математическое моделирование циркуляции 
        океана. –Москва: Наука, 1988. –302 с.
\end{enumerate}

\subsubsection*{Articles}

\begin{enumerate}
    \item Birkhoff G., Diaz J.B. Non-linear network problems // Quart. Appl. 
        Math. -2005. -V. 13, N. 4. -P. 431-443.
    \item Ильин А.М. Разностная схема для дифференциального уравнения с малым 
    параметром при старшей производной // Матем. заметки. -1969. -Т. 6, Вып. 2. 
    -C. 237-248.
\end{enumerate}

\subsubsection*{Materials from Conferences and Similar}

\begin{enumerate}
    \item Osinov V.I. Qualitative study of an extreme problem in pattern 
        recognition theory // Mathematical methods of image recognition: Tr. 
        International conference - Erevan: VC AN Armenia. -1985. -P. 28-34.
\end{enumerate}

\subsubsection*{URLs}

\begin{enumerate}
    \item Ivanov A.A. Programming peculiarities in C, url: 
        \url{http://www.citforum.ru}
\end{enumerate}

\includepdf[pages=-,pagecommand={},width=1.4\textwidth]{title_page.pdf}

\subsection*{Example Abstracts}

In the United States, cardiovascular disease is the number one cause of death 
for both men and women. Heart attacks and strokes can happen because of 
atherosclerosis, or plaque build-up inside arteries, which obstruct blood 
flow to the heart and brain. One common site of atherosclerosis is the carotid
artery bifurcation. This study looks at how the angle between the branching 
arteries of the bifurcation affects the potential for atherosclerosis by 
running flow simulations through virtual models of the bifurcation. The higher
the wall shear stress, turbulence intensity, and turbulence kinetic energy
at the bifurcation, the lower the chance of atherosclerosis. There is an 
optimal angle at which this occurs.\\

В работе предложен новый численный метод для решения квазилинейных 
гиперболических и параболических уравнений. В основу положен модифицированный 
метод характеристик и алгоритм адаптации сетки. Предложенный метод и его 
варианты тестировались на многочисленных модельных задачах. Результаты 
тестирования показали высокую точность предлагаемых алгоритмов, гораздо более 
высокую, чем классические разностные схемы первого и второго порядка 
(Лакса-Вендроффа, Мак-Кормака и т.д.).

\subsection*{Example Conclusions}

When we first started this study, we hypothesized that the narrower the angle 
between the branching arteries, the higher the turbulence intensity, turbulence
kinetic energy, and wall shear stress, and thus the lower the chance of 
atherosclerosis. The first two simulations we ran were on the wide angle and 
average angle models. The data from these two simulations led us to believe 
that our hypothesis was correct. However, when the data came back from the 
simulation on the narrow angle model, it was clear that our hypothesis needed 
to be revised. The highest values of all three measures occurred in the average
angle model, indicating that the least chance of atherosclerosis occurs when
the branching arteries are about 50° apart. This means that instead the lowest 
angle being best, there is an optimal angle at which the three measures are 
highest.\\

В работе предложен новый метод численного решения квазилинейных уравнений
гиперболического и параболического типов, метод условно назван адаптивно-
характеристическим. На его основе построены вычислительные алгоритмы, решающие 
задачу Коши для квазилинейного гиперболического закона сохранения и 
параболического уравнения Бюргерса. Построены многочисленные тестовые примеры, 
позволяющие проиллюстрировать работу предложенного метода.\\
Выполненные теоретические построения, численные эксперименты и анализ 
результатов позволяют сделать следующее заключение.
\begin{enumerate}
    \item Доказана эквивалентность задачи Коши для квазилинейного 
        гиперболического уравнения и краевой задачи для характеристической 
        системы.
    \item Разработан численный метод решения краевой задачи для 
        характеристической системы, который обладает достаточной «гибкостью» 
        для модификации, позволяющей повышать порядок точности метода и решать 
        задачи параболического типа.
    \item Разработанный метод и его модифицированные варианты обладают более 
        высоким порядком точности, чем классические разностные схемы первого и 
        второго порядков (Лакса-Вендроффа, Мак-Кормака, TVD-схема Хартена).
    \item Поскольку составной частью алгоритма является механизм адаптации 
        вычислительной сетки, то одним из его положительных качеств можно 
        считать его небольшую ресурсоемкость: он требует на порядок меньшее 
        количество узлов, чем классические разностные схемы. Повышенный порядок 
        точности обеспечивается за счет сгущения сетки в тех областях, где 
        наблюдаются большие градиенты решения.
\end{enumerate}
Результаты выполненных расчетов и их качественное согласие с экспериментальными 
данными позволяют сделать общий вывод о том, что предложенный адаптивно-
характеристический метод дает хорошее приближение к точному решению. А 
разработанный на его основе программный код может быть рекомендован для 
использования в научных и инженерных расчетах различных задач гидро и 
газодинамики, механики сплошной среды и во многих других областях.

\end{document}
