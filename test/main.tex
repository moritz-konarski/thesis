\documentclass[12pt,a4paper,reqno]{article}

% for each sub file in the same folder
% \documentclass[../main.tex]{subfiles}

\usepackage{../thesis}

\begin{document}
\pagestyle{plain}

\renewcommand{\contentsname}{Table of Contents}
\tableofcontents

\section{Introduction}

\textbf{Lorem ipsum dolor sit amet, consectetur adipiscing elit. Morbi lacinia ante leo,
id tristique turpis semper nec.} Phasellus at feugiat libero. Fusce sed porta quam.
Vivamus bibendum nulla in urna sagittis, vel aliquam neque commodo. Ut a faucibus
arcu, eu placerat ex. Vestibulum nec quam neque. Sed vitae quam sit amet tellus suscipit
aliquam id eget ipsum. Mauris consequat tristique scelerisque. Orci varius natoque
penatibus et magnis dis parturient montes, nascetur ridiculus mus. Donec in tempus nunc,
laoreet pretium sem. Proin commodo, augue eu mollis hendrerit, sapien est lobortis tellus,
mattis convallis tortor lacus a augue. Phasellus euismod aliquet iaculis. Aenean fringilla
euismod lectus sit amet mollis. Proin erat erat, pharetra vel diam in, consectetur hendrerit
odio. Praesent lacinia tellus tempor est consectetur, eget eleifend est
sollicitudin.\par

\subsection{Hello there}

\textit{Nulla pulvinar lectus ac velit suscipit interdum quis at ipsum. Suspendisse in
tincidunt nibh, id facilisis leo. Praesent bibendum, nibh aliquet interdum dapibus, tortor
lorem pellentesque} est, et eleifend velit arcu at eros. Nunc porttitor dui sit amet mollis
hendrerit. Ut mauris elit, ullamcorper sed tempor sit amet, luctus quis purus. Nunc eros
erat, consequat at ligula id, semper interdum massa. Phasellus vestibulum, augue sit amet
accumsan feugiat, ex lectus maximus urna, vel scelerisque leo nunc sed mi. Vestibulum
ante ipsum primis in faucibus orci luctus et ultrices posuere cubilia curae; Nam purus ex,
semper volutpat dolor in, ornare congue enim. Fusce purus mi, feugiat ut nulla in, cursus
facilisis velit.\par

\subsubsection{Methods}

\textbf{\textit{Donec tempor sodales tortor a facilisis. Duis neque est, hendrerit ac lobortis ac,
fermentum non augue. Proin commodo quis nisi at congue. Mauris congue ante non
purus hendrerit aliquet. Sed neque odio, dignissim et maximus interdum, vehicula quis
metus. Praesent}} sagittis tincidunt est, eu tempor dui pharetra vitae. Pellentesque
maximus euismod tellus sed blandit.\par

\section{Method 1}

Vivamus at blandit ligula, vitae aliquam risus. Aliquam a lectus lacinia, ornare dui
vel, ultricies ex. Nulla facilisi. Etiam eu scelerisque ipsum. Class aptent taciti sociosqu ad
litora torquent per conubia nostra, per inceptos himenaeos. Etiam non neque eros. Ut a ex
et nunc iaculis tincidunt. Morbi semper nulla tortor, quis faucibus ipsum
fermentum id.\par

Mauris ipsum quam, rhoncus nec vehicula vitae, viverra et nisi. Nullam
pellentesque non magna sit amet efficitur. Duis efficitur lectus in placerat euismod. Fusce
eu diam odio. Maecenas vehicula dapibus erat, ac vehicula tellus tempus non. Vestibulum
pharetra varius leo, a posuere ante posuere ut. Donec sit amet bibendum nulla, sed
ultricies libero.

Here are some mathematics examples using the standard word font for times new
roman. This shall be used for comparison as well. I will hope that I can replicate this
exactly using LuaLaTex. I hope so.
\begin{equation}\label{eq:test}
    (x+a)^n = \sum_{k=0}^n \binom{n}{k}x^k a^{n-k}
\end{equation}
\begin{equation}\nonumber
    f(x) = a_0 + \sum_{n=1}^\infty 
        \left( a_n \cos \frac{n \pi x}{L} + b_n \sin \frac{n \pi x}{L}\right)
\end{equation}
\begin{equation}\nonumber
    \cos \alpha + \cos \beta = 2 \cos \frac{1}{2} (\alpha + \beta)
        \cos \frac{1}{2} (\alpha - \beta)
\end{equation}

\noindent
In \figref{fig:test} we can see what was indicated by \eqref{eq:test}

\begin{figure}[H]
    \center
    \includegraphics[width=0.8\textwidth]{example-image-a}
    \caption{This is a test caption}
    \label{fig:test}
\end{figure}

\noindent
\tabref{tab:test} may illustrate why \figref{fig:test} looks the way it looks.

\begin{table}[H]
    \center
    \begin{tabular}{| l | l | r |}\hline
        \rowcolor{Gray}
        A & B & C \\\hline
        231123.2134234 & 30491234 & 02341029348 \\\hline
        231123.2134234 & 30491234 & 02341029348 \\\hline
        231123.2134234 & 30491234 & 02341029348 \\\hline
        231123.2134234 & 30491234 & 02341029348 \\\hline
    \end{tabular}
    \caption{This is a test caption for the table}
    \label{tab:test}
\end{table}

\end{document}
