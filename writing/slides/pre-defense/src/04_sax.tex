\documentclass[../01_main.tex]{subfiles}

\begin{document}

\section{Process of SAX, MSAX}

\subsection{Overview}

\begin{frame}{Process of SAX and MSAX}
    \begin{itemize}
        \item SAX and MSAX require the same steps
    \end{itemize}
    \begin{enumerate}
        \item Z-Normalization
        \item Dimensionality Reduction
        \item Discretization
        \item Distance Measure
    \end{enumerate}
\end{frame}

\subsection{Process}

\begin{frame}{Step 1: Z-Normalization}
    \begin{columns}
        \begin{column}{0.5\textwidth}
    \begin{itemize}
        \item assumption: data is approximately normally distributed
        \item to analyze time series, they are first normalized so that $\mu=0$
            and $\sigma=1$
            \begin{equation}
                x^i[t] = \frac{X^i[t] - \mu}{\sigma}
            \end{equation}
        \item enables comparison between different time series
    \end{itemize}
        \end{column}
        \begin{column}{0.5\textwidth}
    \begin{itemize}
    \item perform multivariate normalization
    \item mean vector $\mu$ as vector of the means for each time series
    \item covariance matrix $\Sigma$ for variances and covariances between the
        different time series
        \begin{equation}\nonumber
            \mathbf{x}[t] = \Sigma^{-1/2} (\mathbf{X}[t] - \mu)
        \end{equation}
    \item this uses mean and covariance structure of the multivariate data
    \end{itemize}
        \end{column}
    \end{columns}
\end{frame}

\begin{frame}{Step 2: Dimensionality Reduction}
    \begin{columns}
        \begin{column}{0.5\textwidth}
    \begin{itemize}
        \item piecewise aggregate approximation (PAA) reduces dimensionality 
            (through averaging of segments)
        \item simplifies the time series
        \item results in $\bar C = \bar c_1,\dots,\bar c_w$
        \item getting element $i$ of $\bar C$ (time series $x$ has length $n$)
            \begin{equation}\nonumber
                \bar c_i = \frac{w}{n} \sum_{j = \frac{n}{w}(i-1)+1}^
                    {\frac{n}{w}i}x_j
            \end{equation}
    \end{itemize}
        \end{column}
        \begin{column}{0.5\textwidth}
    \begin{itemize}
    \item PAA is used here like in SAX, each time series is handled
        separately
    \item the discretization process works the same way too
    \item each time series component is discretized separately
    \item to differentiate them, one alphabet can for example be uppercase
    \end{itemize}
        \end{column}
    \end{columns}
\end{frame}

\begin{frame}{PAA Graph}
    \begin{figure}[H]
        \centering
        \fbox{\includegraphics[height=0.73\textheight]{saxpaa}}
        \caption{ECG with PAA (MITBIH/100, $w = 18$, $n = 360$)}
    \end{figure}
\end{frame}

\begin{frame}{Step 3: Discretization}
    \begin{columns}
        \begin{column}{0.5\textwidth}
    \begin{itemize}
        \item assign letters to PAA segments
        \item breakpoints are created that divide a Gaussian curve into equal
            parts
        \item number of breakpoints dependent on size of alphabet
        \item all PAA below lowest breakpoint are \textit{a}, the ones above it
            \textit{b}\dots
    \end{itemize}
        \end{column}
        \begin{column}{0.5\textwidth}
        \end{column}
    \end{columns}
\end{frame}

\begin{frame}{Discretization Graph}
    \begin{figure}[H]
        \centering
        \fbox{\includegraphics[height=0.73\textheight]{sax}}
        \caption{SAX (MITBIH/100, $w = 18$, $n = 360$, alphabet
        size 3)}
    \end{figure}
\end{frame}

\begin{frame}{Step 4: Distance Measure}
    \begin{columns}
        \begin{column}{0.5\textwidth}
    \begin{itemize}
        \item SAX lower bounds the Euclidean distance, i.e. SAX distances
            correspond to Euclidean distances
        \item Euclidean distance between 2 time series $Q,C$
            \begin{equation}\nonumber
                D(Q,C) \equiv \sqrt{\sum_{i=1}^n (q_i - c_1)^2}
            \end{equation}
        \item SAX distance
            \begin{equation}\nonumber
                MINDIST\left(\hat Q,\hat C\right) \equiv \sqrt{\frac{n}{w}}
                \sqrt{\sum_{i=1}^w \left(dist(\hat q_i, \hat c_i)\right)^2}
            \end{equation}
        \item $dist(\hat q_i, \hat c_i)$ is the difference between the 
            breakpoints of $\hat q_i, \hat c_i$
    \end{itemize}
        \end{column}
        \begin{column}{0.5\textwidth}
    \begin{itemize}
    \item this distance measure is based on $MINDIST$
    \item it is also lower bounding the Euclidean distance
    \item it adds an extra step of adding the $dist$ values for the time
        series components
        \begin{equation}\nonumber
            MINDIST\_MSAX(Q,C) = \sqrt{\frac{n}{w}} \sqrt{\sum^w_{i=0}
            \left(\sum_{i=0}^n dist(q[i],c[i])^2\right)}
        \end{equation}
\end{itemize}
        \end{column}
    \end{columns}
\end{frame}

\end{document}
