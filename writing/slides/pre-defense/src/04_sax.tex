\documentclass[../01_main.tex]{subfiles}

\begin{document}

\section{SAX}

\begin{frame}{Z-Normalization}
    \begin{itemize}
        \item assumption: data is approximately normally distributed
        \item to analyze time series, they are first normalized so that $\mu=0$
            and $\sigma=1$
            \begin{equation}\nonumber
                x^i[t] = \frac{X^i[t] - \mu}{\sigma}
            \end{equation}
        \item enables comparison between different time series
    \end{itemize}
\end{frame}

\begin{frame}{PAA}
    \begin{itemize}
        \item piecewise aggregate approximation (PAA) reduces dimensionality 
            (through averaging of segments)
        \item simplifies the time series
        \item results in $\bar C = \bar c_1,\dots,\bar c_w$
        \item getting element $i$ of $\bar C$ (time series $x$ has length $n$)
            \begin{equation}\nonumber
                \bar c_i = \frac{w}{n} \sum_{j = \frac{n}{w}(i-1)+1}^
                    {\frac{n}{w}i}x_j
            \end{equation}
    \end{itemize}
\end{frame}

\begin{frame}{PAA Graph}
    \begin{figure}[H]
        \centering
        \fbox{\includegraphics[width=0.9\textwidth]{saxpaa}}
        \caption{ECG with PAA (MITBIH/100, $w = 18$, $n = 360$)}
    \end{figure}
\end{frame}

\begin{frame}{Discretization}
    \begin{itemize}
        \item assign letters to PAA segments
        \item breakpoints are created that divide a Gaussian curve into equal
            parts
        \item number of breakpoints dependent on size of alphabet
        \item all PAA below lowest breakpoint are \textit{a}, the ones above it
            \textit{b}\dots
    \end{itemize}
\end{frame}

\begin{frame}{Discretization Graph}
    \begin{figure}[H]
        \centering
        \fbox{\includegraphics[width=0.9\textwidth]{sax}}
        \caption{SAX (MITBIH/100, $w = 18$, $n = 360$, alphabet
        size 3)}
    \end{figure}
\end{frame}

\begin{frame}{Distance Measure}
    \begin{itemize}
        \item SAX lower bounds the Euclidean distance, i.e. SAX distances
            correspond to Euclidean distances
        \item Euclidean distance between 2 time series $Q,C$
            \begin{equation}\nonumber
                D(Q,C) \equiv \sqrt{\sum_{i=1}^n (q_i - c_1)^2}
            \end{equation}
        \item SAX distance
            \begin{equation}\nonumber
                MINDIST\left(\hat Q,\hat C\right) \equiv \sqrt{\frac{n}{w}}
                \sqrt{\sum_{i=1}^w \left(dist(\hat q_i, \hat c_i)\right)^2}
            \end{equation}
        \item $dist(\hat q_i, \hat c_i)$ is the difference between the 
            breakpoints of $\hat q_i, \hat c_i$
    \end{itemize}
\end{frame}

\end{document}
