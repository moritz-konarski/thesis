\documentclass[../01_main.tex]{subfiles}

\begin{document}

\section{MSAX}

\begin{frame}{Z-Normalization}
    \begin{itemize}
        \item perform multivariate normalization
        \item mean vector $\mu$ as vector of the means for each time series
        \item covariance matrix $\Sigma$ for variances and covariances between the
            different time series
            \begin{equation}\nonumber
                \mathbf{x}[t] = \Sigma^{-1/2} (\mathbf{X}[t] - \mu)
            \end{equation}
        \item this uses mean and covariance structure of the multivariate data
    \end{itemize}
\end{frame}

\begin{frame}{PAA and Discretization}
    \begin{itemize}
        \item PAA is used here like in SAX, each time series is handled
            separately
        \item the discretization process works the same way too
        \item each time series component is discretized separately
        \item to differentiate them, one alphabet can for example be uppercase
    \end{itemize}
\end{frame}

\begin{frame}{Discretization Graph}
    \begin{figure}
        \fbox{\includegraphics[width=0.9\textwidth]{msax}}
        \caption{MSAX (MITBIH/100, $w = 18$, $n = 360$, alphabet size 3)}
    \end{figure}
\end{frame}

\begin{frame}{Distance Measure}
    \begin{itemize}
        \item this distance measure is based on $MINDIST$
        \item it is also lower bounding the Euclidean distance
        \item it adds an extra step of adding the $dist$ values for the time
            series components
            \begin{equation}\nonumber
                MINDIST\_MSAX(Q,C) = \sqrt{\frac{n}{w}} \sqrt{\sum^w_{i=0}
                \left(\sum_{i=0}^n dist(q[i],c[i])^2\right)}
            \end{equation}
    \end{itemize}
\end{frame}

\end{document}
