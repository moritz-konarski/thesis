\documentclass[../01_main.tex]{subfiles}

\begin{document}

\section{SAX and MSAX}

\subsection*{Process}

\begin{frame}{Step 1: Z-Normalization}
    \mynote{say that the process is the same as MSAX based on SAX}
    \begin{block}{Assumption}
        The time series values are normally distributed.
    \end{block}%
    % TODO: cite this
    \mynote{to compare time series, normalization is the accepted step}
    % TODO: test this in program
    \begin{columns}[t]
        \begin{column}{0.5\textwidth}
            \begin{center} SAX \end{center}
            \begin{itemize}
                \item normalize univariate time series
                \item uses scalar mean and variance
                %\item results in mean of 0 and standard deviation of 1
            \end{itemize}
        \end{column}
        \vrule{}
        \begin{column}{0.5\textwidth}
           \begin{center} MSAX \end{center}
           \begin{itemize}
               \item normalize multivariate time series
                    %TODO: research what exactly this means
               \item uses vector mean and covariance matrix
               \mynote{what is this}
               %\item results in mean of 0 and standard deviation of 1
               % TODO: is this true?
               \mynote{data is now uncorrelated}
               \mynote{takes into account the correlation between leads}
           \end{itemize}
       \end{column}
    \end{columns}
    \vspace{1em}
    \begin{itemize}
        \item[$\Rightarrow$] time series have mean 0 and standard
            deviation 1
    \end{itemize}
\end{frame}

\begin{frame}{Step 2: Dimensionality Reduction}
    \begin{block}{Method}
        Piecewise Aggregate Approximation (PAA) takes $T$ values and finds 
        the averages of $w$ segments ($w<T$), reducing the complexity.
    \end{block}%
    \begin{columns}[t]
        \begin{column}{0.5\textwidth}
            \begin{center} SAX \end{center}
            \begin{itemize}
                \item apply PAA to time series
            \end{itemize}
        \end{column}
        \vrule{}
        \begin{column}{0.5\textwidth}
           \begin{center} MSAX \end{center}
           \begin{itemize}
               \item apply PAA to each of the time series individually
           \end{itemize}
       \end{column}
    \end{columns}
    \vspace{1em}
    \begin{itemize}
        \item[$\Rightarrow$] time series has been simplified, consisting of
            fewer elements
    \end{itemize}
\end{frame}

\begin{frame}[plain]
    \begin{figure}
        \centering
        \fbox{\includegraphics[height=0.73\textheight]{saxpaa}}
        \caption{ECG with PAA (MITBIH/100, $w = 18$, $T = 360$)}
    \end{figure}
\end{frame}

\begin{frame}{Step 3: Discretization}
    \begin{block}{Method}
        Create breakpoints splitting a normal curve into $N$ segments; each
        segment has equal probability. Then assign a letter to each segment;
        \textit{a} to the lowest, \textit{b} to the next\dots{} Result is called
        a \textit{word}.
    \end{block}%
    \begin{columns}[t]
        \begin{column}{0.5\textwidth}
            \begin{center} SAX \end{center}
            \begin{itemize}
                \item discretize the time series
                \item results in one word
            \end{itemize}
        \end{column}
        \vrule{}
        \begin{column}{0.5\textwidth}
           \begin{center} MSAX \end{center}
           \begin{itemize}
               \item discretize each time series individually
               \item results in one word, multiple letters per segment
           \end{itemize}
       \end{column}
    \end{columns}
    \mynote{simplifies time series even more}
    \mynote{creates discrete categories, can be more useful}
\end{frame}

% TODO: add slide with some breakpoints for illustration

\begin{frame}[plain]
    \begin{figure}
        \centering
        \fbox{\includegraphics[height=0.73\textheight]{sax}}
        \caption{SAX (MITBIH/100, $w = 18$, $T = 360$, alphabet
        size 4)}
    \end{figure}
\end{frame}

\begin{frame}{Step 4: Distance Measure}
    \begin{block}{Method}
        To compare two same-length SAX words, a distance measure is needed. 
        Distance is defined for letters: 0 for neighbors; absolute difference 
        of breakpoints otherwise. 
    \end{block}%
    \begin{columns}[t]
        \begin{column}{0.5\textwidth}
            \begin{center} SAX \end{center}
                \begin{equation}\nonumber
                    \sqrt{\frac{T}{w}}\sqrt{\sum_{i=1}^w 
                    \left(dist(\hat q[i], \hat c[i])\right)^2}
                \end{equation}
            \mynote{$n$ -- length of original time series}
            \mynote{$w$ -- length of word}
        \end{column}
        \vrule{}
        \begin{column}{0.5\textwidth}
           \begin{center} MSAX \end{center}
            \begin{equation}\nonumber
                \sqrt{\frac{T}{w}} \sqrt{\sum^w_{i=1}
                \left(\sum_{j=1}^n 
                \left(dist(\hat q_j[i], \hat c_j[i])\right)^2\right)}
            \end{equation}
       \end{column}
    \end{columns}
    \mynote{this lower-bounds the euclidean distance, meaning that results in
    SAX should hold true for the real data too}
\end{frame}

% TODO: add slide with difference matrix and example 

\end{document}
