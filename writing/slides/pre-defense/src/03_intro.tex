\documentclass[../01_main.tex]{subfiles}

\begin{document}

\section{Introduction}

\subsection*{ECG Basics}

\begin{frame}{What is an ECG?}
    \begin{columns}
        \begin{column}{0.55\textwidth}
            \begin{figure}
                \includegraphics[width=\textwidth,
                height=0.7\textheight]{03_ecg}\vspace{-1.1em}
                \caption{ECG of one heartbeat}
            \end{figure}
        \end{column}
        \begin{column}{0.45\textwidth}
            \begin{itemize}[leftmargin=0pt]
                \item electrocardiogram (ECG or EKG) records the heart's 
                    electrical activity 
                    \mynote{muscle contractions caused by electric pulses}
                    \mynote{electric pulse can be measured on the skin}
                \item contains up to 12 simultaneous measurements---the leads
                    \mycite{moody1992}
                    \mynote{the measuring things are called electrodes}
                    \mynote{electrodes form leads (need 2 to measure anything)}
                    \mynote{they have specific positions and names}
                    \mynote{12 leads is the modern standard}
                \item common medical diagnostic tool\mycite{zotero-330}
                    \mynote{most types of heart disease can be detected}
                    \mynote{diagnosis and analysis is performed by trained
                    cardiologists}
                    \mynote{\textbf{datasets available online; contain 2 or 
                    more leads (the most significant ones)}}
                    \mynote{\textbf{I will be using online datasets for my
                    analysis}}
                    \mynote{heart diseases are some of the most deadly ones,
                    thus ECG are really important}
            \end{itemize}
        \end{column}
    \end{columns}
\end{frame}

\begin{frame}[plain]
    \begin{figure}
        \includegraphics[width=\textwidth,height=\textheight]{03_ecg_ann}%
        \vspace{-1em}%
        \caption{Annotated ECG of one heartbeat}
    \end{figure}
    \mynote{P wave: atria depolarizing / blood entering the heart}
    \mynote{QRS complex: ventricular depolarization / heart contraction pumping
    blood}
    \mynote{T: return of ventricle to polarized state}
    \mynote{U: present in 25\%, maybe some feedback}
    \mynote{P wave: atria depolarizing / blood entering the heart}
    \mynote{ST-segment: significant, depression, elevation, slope show
    ischaemia}
    \mynote{R-R interval: shows rhythm and thus arrhytmia etc}
\end{frame}

\begin{frame}{ECGs as Time Series}
  %   \begin{definition}
  %       A discrete multivariate time series is 
  %       $\left\{\mathbf{x}[t] \right\}_{t \in \{1, \dots, T \}}$,
  %       where $t$ is a moment in time, $T$ is the number of measurements,
  %       $\mathbf{x}[t] = \left(x_1[t], \dots, x_n[t] \right)$ is a set of
  %       measurements at moment $t$, and $n$ is the number of measurements at
  %       a moment $t$.
  %   \end{definition}
  %  \begin{definition}
  %      A discrete multivariate time series is a sequence of values
  %      $$\left\{\mathbf{x}[t] \right\}_{t \in \{1, \dots, T \}}$$
  %      where\\
  %      \begin{itemize}[itemsep=0pt]
  %          \item $\mathbf{x}[t] = \left(x_1[t], \dots, x_n[t] \right)$ --- set 
  %              of values at moment $t$,
  %          \item $t$ --- discrete moment in time, 
  %          \item $T$ --- number of sets of values,
  %          \item $n$ --- number of values at moment $t$.
  %      \end{itemize}
  %  \end{definition}
    \begin{definition}
        A discrete time series is an ordered sequence that, at discrete points 
        in time, has $n$ values each. If $n=1$, the series is univariate and 
        if $n>1$, it is multivariate.
    \end{definition}
    \begin{itemize}
        \item digital ECGs are discrete multivariate time series:
            \mycite{anacleto2020}
            \begin{itemize}
                \item have $>1$ value at each point, often $n=12$
                    \mynote{modern ECGs have at least 2, most have 12}
                \item recorded at discrete, evenly spaced time points
                    \mynote{digital ones have set sampling frequencies, even
                    the machines have set frequencies}
            \end{itemize}
            \mynote{multivariate: measure more than 1 lead per time point}
            \mynote{discrete: set sample frequency in the machines}
            \mynote{discrete: because measured at discrete moments in time}
            \mynote{time series: they are data measured at equal time
            intervals}
            \mynote{$n$ measurements per point in time (i.e. leads)}
            \mynote{$n=1$ is univariate, $n>1$ is multivariate}
        \item time series analysis methods can be applied to ECGs
    \end{itemize}
\end{frame}

\subsection*{ECG Analysis}

\begin{frame}{ECG Analysis}
    \begin{itemize}
        \item standard method: manual analysis by cardiologist
            \mynote{is relatively slow; time is of the essence}
            \mynote{lots of training required}
            \mynote{error prone}
            \mynote{maybe not feasible for long ECGs}
        \item recently: automated or computer-assisted ECG analysis
            \mynote{can speed up process}
            \mynote{can pick up details humans miss}
        \item multiple stages:
            \mycite{kligfieldpaul2007}
            \begin{enumerate*}[label=(\arabic*)]
                \item signal acquisition;
                    \mynote{digitizing paper ECGs or recording digital ones}
                \item data transformation, processing, filtering;
                    \mynote{filtering to remove various types of noise}
                    \mynote{reduce complexity of the data}
                \item waveform recognition, feature extraction;
                    \mynote{select important features and neglect irrelevant
                    ones to ease analysis}
                \item classification
                    \mynote{often added, figure out if there is some disease
                    present or not}
            \end{enumerate*}
        \item current research focus: artificial neural networks\dots
            \mycite{xie202}
            \mynote{balance between accuracy and complexity needed}
            \mynote{ann: hand all the steps discussed to a NN; use as good
            classifier too}
        \item relatively new methods are SAX, MSAX, and HOTSAX
    \end{itemize}
\end{frame}

\begin{frame}{SAX, MSAX, and HOTSAX}
    \begin{itemize}
        \item \citeauthor*{lin2003} (\citeyear{lin2003}): Symbolic Aggregate 
            Approximation (SAX)---simplified, symbolic representation
            \mycite{lin2003}
            \mynote{ecg as letters that mean same thing as original}
            \mynote{guaranteed to behave like the original data}
            \mynote{works on univariate time series}
            \mynote{has been used on ECGs}
            \mycite{zhang2019}
        \item \citeauthor*{anacleto2020} (\citeyear{anacleto2020}): Multivariate 
            SAX (MSAX)---expands SAX to multivariate time series
            \mycite{anacleto2020}
            \mynote{takes the correlation between ecg leads into account}
            \mynote{cov mat: covariance between each lead and variance on diag}
        \item \citeauthor*{keogh2005} (\citeyear{keogh2005}): Heuristically 
            Ordered Time series using Symbolic Aggregate Approximation 
            (HOTSAX)---discord discovery algorithm for SAX
            \mynote{uses sax representation to make the finding of discords
            easier}
            \mynote{can use MSAX just as well}
    \end{itemize}
\end{frame}

\begin{frame}{Time Series Discords}
    \begin{definition}
        A time series discord is the subsequence of a time series that is most
        different from all other subsequences.\\
        $k$ time series discords are the $k$ most different subsequences.
    \end{definition}
    \begin{itemize}
        \item discords represent anomalies in an ECG
            \mynote{these can be diseases, noise, etc}
            \mynote{the discord does not discern}
        \item can be found by comparing all subsequences to all other
            subsequences; does not scale well 
            \mycite{keogh2005}
            \mynote{this is not feasible because of complexity}
        \item HOTSAX makes this process faster
    \end{itemize}
\end{frame}

\subsection*{Hypothesis}

\begin{frame}{Hypothesis}
    \begin{exampleblock}{}
        HOTSAX with MSAX will increase the number of relevant discords detected 
        compared to HOTSAX with SAX.\\
        Accuracy can be judged with the help of annotated ECGs from online
        databases.
        \mynote{mention that MSAX to ECGs in particular is new}
        \mynote{mention that HOTSAX with MSAX is new}
        \mynote{results will not be great as HOTSAX is not a real classifier;
        this is about finding out if MSAX adds more useful information}
        \mynote{THIS METHOD WILL NOT BE SUPER ACCURATE; MANY ECG changes
        are relatively small and would get lost in the SAX process}
        \mynote{THE METHOD HAS NO AWARENESS OF MEDICAL RELEVANCE OR ANY OF
        THAT}
        \mynote{NOVEL: MSAX IS SUPER NEW; HAS NOT BEEN APPLIED TO ECGs IN
        THIS WAY; ALSO NOT USED WITH HOTSAX}
    \end{exampleblock}
\end{frame}

\end{document}
