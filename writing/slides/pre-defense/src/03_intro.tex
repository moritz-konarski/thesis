\documentclass[../01_main.tex]{subfiles}

\begin{document}

\section{Introduction}

\subsection{ECG Basics}

\begin{frame}{What is an ECG?}
    \begin{columns}
        \begin{column}{0.5\textwidth}
            \begin{figure}
                \includegraphics[width=1.1\textwidth]{03_ecg}
                \caption{Section of an ECG}
            \end{figure}
        \end{column}
        \begin{column}{0.5\textwidth}
            \begin{itemize}
                \item electrocardiogram (ECG or EKG) records heart's 
                    electrical activity 
                    \mynote{muscle contractions caused by electric pulses}
                    \mynote{electric pulse can be measured on the skin}
                \item takes up to 12 simultaneous\\measurements\mycite{moody1992a}
                    \mynote{the measuring things are called electrodes}
                    \mynote{electrodes form leads (need 2 to measure anything)}
                    \mynote{they have specific positions and names}
                    \mynote{12 leads is the modern standard}
                \item they are very common medical diagnostic tools 
                    \mynote{most types of heart disease can be detected}
                    \mynote{diagnosis and analysis is performed by trained
                    cardiologists}
                    \mynote{datasets available online; contain 2 or more leads 
                    (the most significant ones) }
            \end{itemize}
        \end{column}
    \end{columns}
\end{frame}

\begin{frame}{ECGs as Time Series}
  %   \begin{definition}
  %       A discrete multivariate time series is 
  %       $\left\{\mathbf{x}[t] \right\}_{t \in \{1, \dots, T \}}$,
  %       where $t$ is a moment in time, $T$ is the number of measurements,
  %       $\mathbf{x}[t] = \left(x_1[t], \dots, x_n[t] \right)$ is a set of
  %       measurements at moment $t$, and $n$ is the number of measurements at
  %       a moment $t$.
  %   \end{definition}
    \begin{definition}
        A discrete multivariate time series is a sequence of values
        $$\left\{\mathbf{x}[t] \right\}_{t \in \{1, \dots, T \}}$$
        where\\
        \begin{itemize}[itemsep=0pt]
            \item $\mathbf{x}[t] = \left(x_1[t], \dots, x_n[t] \right)$ --- set 
                of values at moment $t$,
            \item $t$ --- discrete moment in time, 
            \item $T$ --- number of sets of values,
            \item $n$ --- number of values at moment $t$.
        \end{itemize}
    \end{definition}
    \begin{itemize}
        \item[$\Rightarrow$] ECGs are discrete multivariate time series
            \mycite{anacleto2020}
            \mynote{multivariate: measure more than 1 lead per time point}
            \mynote{discrete: set sample frequency in the machines}
            \mynote{discrete: because measured at discrete moments in time}
            \mynote{time series: they are data measured at equal time
            intervals}
            \mynote{$n$ measurements per point in time (i.e. leads)}
            \mynote{$n=1$ is univariate, $n>1$ is multivariate}
    \end{itemize}
\end{frame}

%\begin{frame}
%\end{frame}

\begin{frame}
    \begin{figure}
        \centering
        \fbox{\includegraphics[width=\textwidth]{wasilewski2012}}
        \caption{An annotated model ECG \mycite{wasilewski2012a}}
    \end{figure}
\end{frame}

\subsection*{ECG Analysis}

\begin{frame}{Automated ECG Analysis}
    \begin{itemize}
        \item ECGs represent large amounts of data, thorough analysis is
            required
        \item 5 stages: (1) signal acquisition, filtering; (2) data 
            transformation, processing; (3) waveform recognition; (4) feature 
            extraction; (5) classification\mycite{kligfieldpaul2007}
        \item some methods include FFT, DWT, ANN, kNN, filters
        \item balance between accuracy and complexity needed
    \end{itemize}
\end{frame}

\begin{frame}{SAX and MSAX}
    \begin{itemize}
        \item Symbolic Aggregate
            Approximation (SAX) creates a simplified, symbolic representation
            \mycite{lin2003}
        \item is guaranteed to behave like the original data
        \item works on univariate time series, has been used on
            ECGs\mycite{zhang2019}
        \item Multivariate SAX (MSAX)\mycite{anacleto2020} expands SAX to 
            multivariate time series
        \item[$\rightarrow$] using MSAX on ECGs should increase the accuracy of
            discord detection compared to SAX
    \end{itemize}
\end{frame}

\end{document}
