\documentclass[../01_main.tex]{subfiles}

\begin{document}

\section{Methods}

\subsection*{SAX and MSAX}

\begin{frame}{Step 1: Z-Normalization}
    \mynote{say that the process is the same as MSAX based on SAX}
    \begin{block}{Assumption}
        The time series values are normally distributed.
    \end{block}%
    \mynote{this is assumed and this worked for other people who applied SAX to
    ECGs}
    \mynote{to compare time series, normalization is the accepted step}
    \begin{columns}[t]
        \begin{column}{0.5\textwidth}
            \begin{center} SAX \end{center}
            \begin{itemize}
                \item normalize univariate time series
                \item uses scalar mean and variance
            \end{itemize}
        \end{column}
        \vrule{}
        \begin{column}{0.5\textwidth}
           \begin{center} MSAX \end{center}
           \begin{itemize}
               \item normalize multivariate time series
               \item uses vector mean and covariance matrix
               \mynote{what is this}
               \mynote{takes into account the correlation between leads}
           \end{itemize}
       \end{column}
    \end{columns}
\end{frame}

\begin{frame}{Step 2: Dimensionality Reduction}
    \begin{block}{PAA}
        Piecewise Aggregate Approximation (PAA) takes $T$ time series points,
        splits it into $w$ ($w<T$) segments, and averages each of them.
    \end{block}
    \mynote{this reduces complexity}
    \begin{columns}[t]
        \begin{column}{0.5\textwidth}
            \begin{center} SAX \end{center}
            \begin{itemize}
                \item apply PAA to time series
            \end{itemize}
        \end{column}
        \vrule{}
        \begin{column}{0.5\textwidth}
           \begin{center} MSAX \end{center}
           \begin{itemize}
               \item apply PAA to each of the time series individually
           \end{itemize}
       \end{column}
    \end{columns}
    \mynote{PAA form of time series is shorter and simpler}
    \mynote{it still somewhat corresponds to the original}
\end{frame}

\begin{frame}[plain]
    \begin{figure}
        \centering
        \fbox{\includegraphics[height=0.73\textheight]{saxpaa}}
        \caption{ECG with PAA (MITBIH/100, $w = 18$, $T = 360$)}
    \end{figure}
\end{frame}

\begin{frame}{Step 3: Discretization}
    \begin{block}{SAX Discretization}
        Find breakpoints splitting $\mathcal{N}(0,1)$ into $B$ equiprobable 
        segments.\\
        Assign a letter to each area: \textit{a} to most-negative, 
        \textit{b} to the next biggest\dots{}\\
        PAA segments get letters based on which area they are in.\\
    \end{block}
    \mynote{result is called word}
    \mynote{$N$ is the alphabet size}
    \mynote{big thing here is that this gives defined probability to each
    letter; makes no sense for real numbers (like PAA values)}
    \begin{columns}[t]
        \begin{column}{0.5\textwidth}
            \begin{center} SAX \end{center}
            \begin{itemize}
                \item discretize the time series
                \item results in one word
            \end{itemize}
        \end{column}
        \vrule{}
        \begin{column}{0.5\textwidth}
           \begin{center} MSAX \end{center}
           \begin{itemize}
               \item discretize each time series individually
               \item results in one word, one letter per time series
           \end{itemize}
       \end{column}
    \end{columns}
    \mynote{simplifies time series even more}
    \mynote{creates discrete categories, can be more useful}
\end{frame}

% TODO: add slide with some breakpoints for illustration
% TODO: add graph that has MSAX series for illustration

\begin{frame}[plain]
    \begin{figure}
        \centering
        \fbox{\includegraphics[height=0.73\textheight]{sax}}
        \caption{SAX (MITBIH/100, $w = 18$, $T = 360$, alphabet
        size 4)}
    \end{figure}
\end{frame}

\begin{frame}{Step 4: Distance Measure}
    \begin{block}{MINDIST}
        A distance measure is defined to compare two SAX words.
        Distance is defined for a pair of letters: 0 if they are neighbors; 
        absolute difference of breakpoint values otherwise. 
    \end{block}%
    \begin{columns}[t]
        \begin{column}{0.5\textwidth}
            \begin{center} SAX \end{center}
                \begin{equation}\nonumber
                    \sqrt{\frac{T}{w}}\sqrt{\sum_{i=1}^w 
                    \left(dist(\hat q[i], \hat c[i])\right)^2}
                \end{equation}
            \mynote{$n$ -- length of original time series}
            \mynote{$w$ -- length of word}
        \end{column}
        \vrule{}
        \begin{column}{0.5\textwidth}
           \begin{center} MSAX \end{center}
            \begin{equation}\nonumber
                \sqrt{\frac{T}{w}} \sqrt{\sum^w_{i=1}
                \left(\sum_{j=1}^n 
                \left(dist(\hat q_j[i], \hat c_j[i])\right)^2\right)}
            \end{equation}
       \end{column}
    \end{columns}
    \mynote{this lower-bounds the euclidean distance, meaning that results in
    SAX should hold true for the real data too}
\end{frame}

% TODO: add slide with difference matrix and example

\subsection*{HOTSAX}

\begin{frame}{HOTSAX}
    \begin{itemize}                                                             
        \item ``brute-force" discord discovery is slow, needs $T^2$ operations
        \item HOTSAX speeds up discord discovery by considering that
            \begin{itemize}
                \item discords are rare, start with rarest segment
                \item similar segments have similar distances, consider together
            \end{itemize}
        \item HOTSAX detects anomalies, it is not a classifier
            \mynote{this is the basic idea that can speed up the process}       
            \mynote{it is not guaranteed to do so, but it does not decrease     
            efficiency}
        \item it uses SAX and MSAX for dimensionality reduction
            \mynote{this speeds up the process even more as we have fewer
            elements}
            \mynote{because of lower bounding, it still gives accurate results}
    \end{itemize}                                                               
\end{frame}                                                                     
                                                                                
% TODO: insert one frame about programming                                      

\end{document}
