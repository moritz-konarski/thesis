\documentclass[../01_main.tex]{subfiles}

\begin{document}

\section{Methods}

\subsection*{Overview}

\begin{frame}{Previous Research}
    \begin{itemize}
        \item \citeauthor*{lin2003} (\citeyear{lin2003}):\\
            Symbolic Aggregate Approximation (SAX)---simplified, symbolic
            representation
            \mycite{lin2003}
            \mynote{ecg as letters that mean same thing as original}
            \mynote{guaranteed to behave like the original data}
            \mynote{works on univariate time series}
            \mynote{has been used on ECGs}
            \mycite{zhang2019}
        \item \citeauthor*{keogh2005} (\citeyear{keogh2005}):\\
            Heuristically Ordered Time series using SAX (HOT SAX)---discord
            discovery algorithm using SAX\mycite{keogh2005}
            \mynote{uses sax representation to make the finding of discords
            easier}
            \mynote{can use MSAX just as well}
        \item \citeauthor*{anacleto2020} (\citeyear{anacleto2020}):\\
            Multivariate SAX (MSAX)---expands SAX to multivariate time series
            \mycite{anacleto2020}
            \mynote{takes the correlation between ecg leads into account}
            \mynote{cov mat: covariance between each lead and variance on diag}
    \end{itemize}
\end{frame}

\begin{frame}{This Work's Novel Contributions}
    \begin{itemize}
        \item application of MSAX to ECG discord discovery
        \item the HOT MSAX algorithm
        \item the expansion of HOT SAX to multivariate time series through HOT
            MSAX
    \end{itemize}
\end{frame}

\subsection*{SAX and MSAX}

\begin{frame}{SAX and MSAX -- Overview}
    \centering
    \begin{tabular}{ p{0.45\textwidth} | p{0.45\textwidth} }
        \multicolumn{1}{c|}{SAX}    & \multicolumn{1}{c}{MSAX}      \\\hline
        \multicolumn{2}{c}{Application} \\\hline
        univariate time series      & multivariate time series      \\
        e.g. a single ECG lead      & e.g. multiple ECG leads       \\\hline
        \multicolumn{2}{c}{Steps}                                   \\\hline
        (1) univariate z-normalization  & (1) multivariate z-normalization  \\
        (2) PAA dimension reduction     & (2) PAA dimension reduction       \\
        (3) SAX discretization          & (3) SAX discretization            \\
    \end{tabular}
\end{frame}

\begin{frame}{SAX and MSAX -- Step (2)}
    \begin{figure}
        \begin{minipage}[c]{0.8\textwidth}
            \includegraphics[height=0.8\textheight]{paa_graph}
        \end{minipage}\hfill
        \begin{minipage}[c]{0.17\textwidth}
            \caption{ECG with PAA (MIT-BIH/103, $w = 18$, $T = 360$)}
        \end{minipage}
    \end{figure}
\end{frame}

\begin{frame}{SAX and MSAX -- Step (3)}
    \begin{figure}
        \begin{minipage}[c]{0.8\textwidth}
            \includegraphics[height=0.8\textheight]{sax_graph}
        \end{minipage}\hfill
        \begin{minipage}[c]{0.17\textwidth}
            \caption{ECG with SAX (MIT-BIH/103, $w = 18$, $T = 360$)}
        \end{minipage}
    \end{figure}
\end{frame}

\begin{frame}{SAX and MSAX -- Distance Measure}
    \begin{itemize}
        \item basically the sum of distances between symbols
        \item symbol distance is based on difference of breakpoints
        \item lower-bounds Euclidean Distance, corresponds to ``real" distance
    \end{itemize}
    \begin{table}
        \caption{Distance matrix for 5 symbols.}
    \centering
    \begin{tabular}{|>{\columncolor{tablegray}}c | r | r | r | r | r |}\hline
        \rowcolor{tablegray} & \multicolumn{1}{c|}{a} & \multicolumn{1}{c|}{b}
            & \multicolumn{1}{c|}{c} & \multicolumn{1}{c|}{d} &
            \multicolumn{1}{c|}{e} \\\hline
        a & 0       & 0        &0.59     &1.09     &1.68  \\\hline
        b & 0       & 0        &0        &0.51     &1.09  \\\hline
        c & 0.59    & 0        &0        &0        &0.59  \\\hline
        d & 1.09    & 0.51     &0        &0        &0     \\\hline
        e & 1.68    & 1.09     &0.59     &0        &0     \\\hline
    \end{tabular}
    \end{table}
\end{frame}

\subsection*{HOT SAX and HOT MSAX}

\begin{frame}{HOT SAX and HOT MSAX -- Overview}
    \begin{itemize}
        \item HOT SAX: find discords in SAX-represented time series
        \item speeds up the ``brute force" approach
        \item classifies time series segments into ``discord" and
            ``non-discord"
        \item HOT MSAX: uses MSAX instead of SAX
        \item HOT MSAX can work with multivariate time series
    \end{itemize}
\end{frame}

\begin{frame}{HOT SAX and HOT MSAX -- Heuristic}
    \begin{itemize}
        \item two assumptions:
            \begin{itemize}[label=--]
                \item time series discords are rare
                \item similar segments are similarly rare
            \end{itemize}
        \item speed up discord discovery:
            \begin{itemize}[label=--]
                \item consider rarest segments first
                \item consider segments similar to the rarest first
            \end{itemize}
    \end{itemize}
\end{frame}

\subsection*{Parameter Optimization}

\begin{frame}{Parameter Optimization}
    \begin{itemize}
        \item perform HOT SAX and HOT MSAX for many parameter combinations
        \item find recall for each combination
        \item recall threshold of 95\%, then sort by precision
        \item choose top 10 of those parameters for each method
        \item choose best parameters using box plot, interquartile range,
            outliers
    \end{itemize}
\end{frame}

\end{document}

\begin{frame}{HOTSAX}
    \begin{itemize}                                                             
        \item ``brute-force" discord discovery is slow, needs $T^2$ operations
        \item HOTSAX speeds up discord discovery by considering:
            \begin{itemize}
                \item discords are rare, start with rarest segment
                \item similar segments have similar distances, consider together
            \end{itemize}
        \item HOTSAX detects anomalies, it is not a classifier
            \mynote{this is the basic idea that can speed up the process}       
            \mynote{it is not guaranteed to do so, but it does not decrease     
            efficiency}
        \item it uses SAX and MSAX for dimensionality reduction
            \mynote{this speeds up the process even more as we have fewer
            elements}
            \mynote{because of lower bounding, it still gives accurate results}
    \end{itemize}                                                               
\end{frame}                                                                     
                                                                                

\subsection*{SAX and MSAX}

\begin{frame}{Step 1: Z-Normalization}
    \mynote{say that the process is the same as MSAX based on SAX}
    \begin{block}{Assumption}
        The time series values are normally distributed.
    \end{block}%
    \mynote{this is assumed and this worked for other people who applied SAX to
    ECGs}
    \mynote{to compare time series, normalization is the accepted step}
    \begin{columns}[t]
        \begin{column}{0.5\textwidth}
            \begin{center} SAX \end{center}
            \begin{itemize}
                \item normalize univariate time series
                \item uses scalar mean and variance
            \end{itemize}
        \end{column}
        \vrule{}
        \begin{column}{0.5\textwidth}
           \begin{center} MSAX \end{center}
           \begin{itemize}
               \item normalize multivariate time series
               \item uses vector mean and covariance matrix
               \mynote{what is this}
               \mynote{takes into account the correlation between leads}
           \end{itemize}
       \end{column}
    \end{columns}
\end{frame}

\begin{frame}{Step 2: Dimensionality Reduction}
    \begin{block}{PAA}
        Piecewise Aggregate Approximation (PAA) takes $T$ time series points,
        splits them into $w$ ($w<T$) segments, and averages each of them.
    \end{block}
    \mynote{this reduces complexity}
    \begin{columns}[t]
        \begin{column}{0.5\textwidth}
            \begin{center} SAX \end{center}
            \begin{itemize}
                \item apply PAA to time series
            \end{itemize}
        \end{column}
        \vrule{}
        \begin{column}{0.5\textwidth}
           \begin{center} MSAX \end{center}
           \begin{itemize}
               \item apply PAA to each of the time series individually
           \end{itemize}
       \end{column}
    \end{columns}
    \mynote{PAA form of time series is shorter and simpler}
    \mynote{it still somewhat corresponds to the original}
\end{frame}

\begin{frame}[plain]
    \begin{figure}
        \centering
        \fbox{\includegraphics[height=0.9\textheight]{paa_graph}}
        \caption{ECG with PAA (MITBIH/100, $w = 18$, $T = 360$)}
    \end{figure}
\end{frame}

\begin{frame}{Step 3: Discretization}
    \begin{block}{SAX Discretization}
        Find breakpoints splitting $\mathcal{N}(0,1)$ into $B$ equiprobable 
        segments.\\
        Assign a letter to each area, starting with \textit{a} to the left-most
        segment.\\
        PAA segments get letters based on which area they are in.\\
    \end{block}
    \mynote{result is called word}
    \mynote{$N$ is the alphabet size}
    \mynote{big thing here is that this gives defined probability to each
    letter; makes no sense for real numbers (like PAA values)}
    \begin{columns}[t]
        \begin{column}{0.5\textwidth}
            \begin{center} SAX \end{center}
            \begin{itemize}
                \item discretize the time series
                \item results in one \emph{word}
            \end{itemize}
        \end{column}
        \vrule{}
        \begin{column}{0.5\textwidth}
           \begin{center} MSAX \end{center}
           \begin{itemize}
               \item discretize each time series
               \item results in one \emph{word} with one letter for each time 
                   series
           \end{itemize}
       \end{column}
    \end{columns}
    \mynote{simplifies time series even more}
    \mynote{creates discrete categories, can be more useful}
\end{frame}

\begin{frame}[plain]
    \begin{figure}
        \centering
        \fbox{\includegraphics[height=0.9\textheight]{sax_graph}}
        \caption{SAX (MITBIH/100, $w = 18$, $T = 360$, $B=4$)}
    \end{figure}
\end{frame}

\begin{frame}{Step 4: Distance Measure}
    \begin{block}{MINDIST}
        A distance measure is defined to compare two SAX words.
        It is defined for a pair of letters, distances between words are sums
        of distances between letters. 
    \end{block}%
    \mynote{distance is based on letter pairs}
    \mynote{SAX: sqrt of sum of squared distance}
    \mynote{MSAX: sqrt of sum of squared distance; also sum all leads}
    \mynote{this lower-bounds the euclidean distance, meaning that results in
    SAX should hold true for the real data too}
    \begin{table}
        \centering
        \caption{Difference matrix for $B=4$}
        \begin{tabular}{| c | c | c | c | c |}\hline
                &   a   &   b   & c     &       d \\\hline
             a  &  0     &0     &0.67449& 1.34898 \\\hline
             b  &  0     &0     &0    & 0.67449\\\hline
             c  &  0.67449 &0     &0    & 0\\\hline
             d  &  1.34898 &0.67449 &0    & 0\\\hline
        \end{tabular}
    \end{table}
\end{frame}
