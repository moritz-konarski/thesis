\documentclass[../01_main.tex]{subfiles}

\begin{document}

\section{Introduction and Background}

\subsection*{Introduction}

% TODO: merge first two slides
\begin{frame}{Introduction}
    \begin{itemize}
        \item ischemic heart disease (IHD) makes up 16\% of global deaths but
            can be diagnosed using an electrocardiogram (ECG or EKG) 
            \mycite{who2020}
        \item manual ECG analysis is slow and error-prone, computers can help
            \mynote{requires years of training and practice to become good at}
        \item long ECGs are problematic even for computers →
            simplification through representation
            \mynote{computers and programs can be faster and more accurate than
            humans}
        \item such representations should be simpler, but still correspond to
            the original data
    \end{itemize}
\end{frame}

\begin{frame}{What is an ECG?}
    \begin{columns}
        \begin{column}{0.60\textwidth}
            \begin{figure}
                \includegraphics[width=\textwidth,
                height=0.73\textheight]{03_ecg}\vspace{-1.1em}
                \caption{ECG of one heartbeat.}
            \end{figure}
            \mynote{P wave: blood entering the heart}
            \mynote{QRS complex: heart contraction pumping blood}
            \mynote{T: return of ventricle to polarized state}
            \mynote{U: present in 25\%}
        \end{column}
        \begin{column}{0.40\textwidth}
            \begin{itemize}[leftmargin=0pt]
                \item records the heart's electrical activity 
                    \mynote{muscle contractions caused by electric pulses}
                    \mynote{electric pulse can be measured on the skin}
                \item contains up to 12 leads (simultaneous measurements)
                    \mycite{moody1992}
                    \mynote{electrodes form leads (need 2 to measure anything)}
                    \mynote{most types of heart disease can be detected}
                    \mynote{diagnosis and analysis is performed by trained
                    cardiologists}
                    \mynote{\textbf{datasets available online; contain 2 or 
                    more leads (the most significant ones)}}
                    \mynote{\textbf{I will be using online datasets for my
                    analysis}}
                    \mynote{heart diseases are some of the most deadly ones,
                    thus ECG are really important}
            \end{itemize}
        \end{column}
    \end{columns}
\end{frame}

\begin{frame}{Representation and Classification}
    \begin{itemize}
        \item for simplification, time series can be represented, e.g. as SAX
            or MSAX
        \item then, the representation can be analyzed instead of the raw data
        \item HOT SAX can be used to classify ECG segments into discord and
            non-discord
        \item this work uses HOT MSAX to combine MSAX and HOT SAX
        \item effectiveness of the methods will be judged by recall and
            precision
    \end{itemize}
\end{frame}

\subsection*{Hypothesis}

\begin{frame}{Research Questions \& Hypothesis}
    \begin{itemize}
        \item Using the MIT-BIH ECG database, what parameters
            maximize HOT SAX and HOT MSAX recall?
        \item Which is better: optimal HOT SAX or optimal HOT MSAX?
    \end{itemize}
    \begin{exampleblock}{}
    HOT MSAX should have higher recall than HOT SAX if both use their 
    best parameters
    \end{exampleblock}
\end{frame}

% TODO: put into conclusion
\begin{frame}{Novel Contributions}
    \begin{itemize}
        \item application of MSAX to ECG discord discovery and medical data in
            general
        \item the HOT MSAX algorithm, a modification of HOT SAX that uses MSAX
        \item the expansion of HOT SAX to multivariate time series through HOT
            MSAX
    \end{itemize}
\end{frame}

\end{document}
