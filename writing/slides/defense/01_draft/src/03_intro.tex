\documentclass[../01_main.tex]{subfiles}

\begin{document}

\section{Introduction and Background}

\subsection*{Introduction}

\begin{frame}{Motivation and Relevance}
    \begin{itemize}
        \item heart diseases kill more people a year than any other disease
        \item ischemic heart disease (IHD) makes up 16\% of global deaths
        \item IHD can be diagnosed using a stress test and electrocardiogram\\
            (ECG or EKG)
            \mynote{ECG changes when condition of the heart changes, this is
            how it can be detected}
        \item manual ECG analysis is slow and error-prone
            \mynote{requires years of training and practice to become good at}
        \item[$\rightarrow$] computerized ECG analysis can help
            \mynote{computers and programs can be faster and more accurate than
            humans}
    \end{itemize}
\end{frame}

\begin{frame}{What is an ECG?}
    \begin{columns}
        \begin{column}{0.65\textwidth}
            \begin{figure}
                \includegraphics[width=\textwidth,
                height=0.73\textheight]{03_ecg_ann}\vspace{-1.1em}
                \caption{Annotated ECG of one heartbeat}
            \end{figure}
            \mynote{P wave: blood entering the heart}
            \mynote{QRS complex: heart contraction pumping blood}
            \mynote{T: return of ventricle to polarized state}
            \mynote{U: present in 25\%}
        \end{column}
        \begin{column}{0.35\textwidth}
            \begin{itemize}[leftmargin=0pt]
                \item records the heart's electrical activity 
                    \mynote{muscle contractions caused by electric pulses}
                    \mynote{electric pulse can be measured on the skin}
                \item contains up to 12 leads (simultaneous measurements)
                    \mycite{moody1992}
                    \mynote{electrodes form leads (need 2 to measure anything)}
                    \mynote{most types of heart disease can be detected}
                    \mynote{diagnosis and analysis is performed by trained
                    cardiologists}
                    \mynote{\textbf{datasets available online; contain 2 or 
                    more leads (the most significant ones)}}
                    \mynote{\textbf{I will be using online datasets for my
                    analysis}}
                    \mynote{heart diseases are some of the most deadly ones,
                    thus ECG are really important}
            \end{itemize}
        \end{column}
    \end{columns}
\end{frame}

\subsection*{Computerized ECG Analysis}

\begin{frame}{Steps of ECG Analysis I}
    General steps:
    \begin{enumerate}[label=(\arabic*)]
        \item signal acquisition and filtering
        \item data transformation or preparation for processing
        \item waveform recognition
        \item feature extraction
        \item classification or diagnosis
    \end{enumerate}
\end{frame}

\begin{frame}{Steps of ECG Analysis II}
    This work's focus:
    \begin{enumerate}[label=(\arabic*)]
        \item signal acquisition and filtering
    \end{enumerate}
    \begin{enumerate}[resume,label=$\rightarrow$]
        \item \textbf{data transformation} or preparation for processing
    \end{enumerate}
    \mynote{reduction in complexity and size of the data}
    \mynote{time series representation methods}
    \mynote{symbolic aggregate approximation to reduce length and complexity}
    \mynote{MSAX as multivariate version of SAX}
    \begin{enumerate}[resume,label=(\arabic*)]
        \item waveform recognition
        \item feature extraction
    \end{enumerate}
    \begin{enumerate}[resume,label=$\rightarrow$]
        \item \textbf{classification} or diagnosis
    \end{enumerate}
    \mynote{assigning of labels to unlabeled data}
    \mynote{cardiologists to medical classification: which disease}
    \mynote{computers often to binary classification: sick / not sick}
    \mynote{HOT SAX is algorithm that does discord / not discord}
    \mynote{can be used to find discords in ECGs, preselection of segments for
    cardiologist}
\end{frame}

\subsection*{Hypothesis}

\begin{frame}{Research Questions \& Hypothesis}
    \begin{itemize}
        \item using the MIT-BIH ECG database, determine the parameters
            maximizing HOT SAX and HOT MSAX recall
        \item compare recall value for best parameters
        \item[$\rightarrow$] HOT MSAX should have higher recall than HOT SAX
            if both use their best parameters
    \end{itemize}
\end{frame}

 % \begin{frame}{ECGs as Time Series}
 %     \begin{definition}
 %         A discrete time series is an ordered sequence which, at discrete points 
 %         in time, has $n$ values each. If $n=1$, the series is univariate and 
 %         if $n>1$, it is multivariate.
 %     \end{definition}
 %     \begin{itemize}
 %         \item digital ECGs are discrete multivariate time series:
 %             \mycite{anacleto2020}
 %             \begin{itemize}
 %                 \item have $>1$ value at each point, often $n=12$
 %                     \mynote{modern ECGs have at least 2, most have 12}
 %                 \item recorded at discrete, evenly spaced time points
 %                     \mynote{digital ones have set sampling frequencies, even
 %                     the machines have set frequencies}
 %             \end{itemize}
 %             \mynote{multivariate: measure more than 1 lead per time point}
 %             \mynote{discrete: set sample frequency in the machines}
 %             \mynote{discrete: because measured at discrete moments in time}
 %             \mynote{time series: they are data measured at equal time
 %             intervals}
 %             \mynote{$n$ measurements per point in time (i.e. leads)}
 %             \mynote{$n=1$ is univariate, $n>1$ is multivariate}
 %         \item time series analysis methods can be applied to ECGs
 %     \end{itemize}
 % \end{frame}

 % \subsection*{ECG Analysis}
 % 
 % \begin{frame}{ECG Analysis}
 %     \begin{itemize}
 %         \item standard method: manual analysis by cardiologist
 %             \mynote{is relatively slow; time is of the essence}
 %             \mynote{lots of training required}
 %             \mynote{error prone}
 %             \mynote{maybe not feasible for long ECGs}
 %         \item automated or computer-assisted ECG analysis seeks to replace that
 %             \mynote{can speed up process}
 %             \mynote{can pick up details humans miss}
 %         \item multiple stages:
 %             \mycite{kligfieldpaul2007}
 %             \begin{enumerate*}[label=(\arabic*)]
 %                 \item signal acquisition;
 %                     \mynote{digitizing paper ECGs or recording digital ones}
 %                 \item data transformation, processing, filtering;
 %                     \mynote{filtering to remove various types of noise}
 %                     \mynote{reduce complexity of the data}
 %                 \item waveform recognition, feature extraction;
 %                     \mynote{select important features and neglect irrelevant
 %                     ones to ease analysis}
 %                 \item classification
 %                     \mynote{often added, figure out if there is some disease
 %                     present or not}
 %             \end{enumerate*}
 %         \item current research focus: artificial neural networks
 %             \mycite{xie2020}
 %             \mynote{balance between accuracy and complexity needed}
 %             \mynote{ann: hand all the steps discussed to a NN; use as good
 %             classifier too}
 %     \end{itemize}
 % \end{frame}
 % 
 % \begin{frame}{SAX, MSAX, and HOTSAX}
 %     \begin{itemize}
 %         \item \citeauthor*{lin2003} (\citeyear{lin2003}):\\
 %             Symbolic Aggregate Approximation (SAX)---simplified, symbolic 
 %             representation
 %             \mycite{lin2003}
 %             \mynote{ecg as letters that mean same thing as original}
 %             \mynote{guaranteed to behave like the original data}
 %             \mynote{works on univariate time series}
 %             \mynote{has been used on ECGs}
 %             \mycite{zhang2019}
 %         \item \citeauthor*{anacleto2020} (\citeyear{anacleto2020}):\\ 
 %             Multivariate SAX (MSAX)---expands SAX to multivariate time series
 %             \mycite{anacleto2020}
 %             \mynote{takes the correlation between ecg leads into account}
 %             \mynote{cov mat: covariance between each lead and variance on diag}
 %         \item \citeauthor*{keogh2005} (\citeyear{keogh2005}):\\ 
 %             Heuristically Ordered Time series using SAX (HOTSAX)---discord 
 %             discovery algorithm for SAX
 %             \mynote{uses sax representation to make the finding of discords
 %             easier}
 %             \mynote{can use MSAX just as well}
 %     \end{itemize}
 % \end{frame}
 % 
 % \begin{frame}{Time Series Discords}
 %     \begin{definition}
 %         A time series discord is the subsequence of a time series that is most
 %         different from all other subsequences.\\
 %         $k$ time series discords are the $k$ most different subsequences.
 %     \end{definition}
 %     \begin{itemize}
 %         \item discords represent anomalies in an ECG
 %             \mynote{these can be diseases, noise, etc}
 %             \mynote{the discord does not discern}
 %         \item HOTSAX enables fast discord discovery
 %             \mycite{keogh2005}
 %     \end{itemize}
 % \end{frame}
 % 
 % \subsection*{Hypothesis}
 % 
 % \begin{frame}{Hypothesis}
 %     \begin{exampleblock}{}
 %         HOTSAX with MSAX will increase the number of relevant discords detected 
 %         compared to HOTSAX with SAX.
 %     \end{exampleblock}
 %     \mynote{\textbf{mention that MSAX to ECGs in particular is new}}
 %     \mynote{\textbf{mention that HOTSAX with MSAX is new}}
 %     \mynote{THIS METHOD WILL NOT BE SUPER ACCURATE; MANY ECG changes
 %     are relatively small and would get lost in the SAX process}
 % \end{frame}

\end{document}
