\documentclass[../01_main.tex]{subfiles}

\begin{document}

\section{Appendix}

\begin{frame}{}
    \vfill
    \centering
    \begin{beamercolorbox}[sep=8pt,center]{title}
        \usebeamerfont{title}Appendix\par%
    \end{beamercolorbox}
    \vfill
\end{frame}

\begin{frame}{Annotated ECG Graph}
    \label{s-ann-ecg}
    \centering
    \includegraphics[width=\textwidth,height=0.85\textheight]{03_ecg_ann}
\end{frame}

\begin{frame}{Background on Methods}
    \label{s-method-bg}
    \begin{itemize}
        \item \citeauthor*{lin2003} (\citeyear{lin2003}):\\
            Symbolic Aggregate Approximation (SAX): simplified, symbolic
            representation
            \mycite{lin2003}
            \mycite{zhang2019}
        \item \citeauthor*{keogh2005} (\citeyear{keogh2005}):\\
            Heuristically Ordered Time series using SAX (HOT SAX): discord
            discovery algorithm using SAX\mycite{keogh2005}
        \item \citeauthor*{anacleto2020} (\citeyear{anacleto2020}):\\
            Multivariate SAX (MSAX): expands SAX to multivariate time series
            \mycite{anacleto2020}
    \end{itemize}
\end{frame}

\begin{frame}{Time Series}
    \label{s-ts}
    Time series after \cite{anacleto2020}:
    \begin{equation}\nonumber
        \left\{ \mathbf{v} [t] \right\}_{t \in \{1, \ldots, T\}}
    \end{equation}
    is a $n$-variate time series where for each time point $t$,
    \begin{equation}\nonumber
        \mathbf{v} [t] = \left(v_1[t], \ldots, v_n[t] \right)^T
    \end{equation}
    are the values. $T$ is length of time series, $n$ is the number of values 
    per moment
\end{frame}

\begin{frame}{Univariate Normalization}
    \label{s-u-norm}
    Sample mean:
    \begin{equation}\nonumber
        \overline{x} = \frac{1}{T} \sum_{t=1}^T{\mathbf{v}[t]}.
    \end{equation}
    Sample standard deviation:
    \begin{equation}\nonumber
        s = \sqrt{\frac{1}{T-1} \sum_{t=1}^T{\left(\mathbf{v}[t] - \overline{x} 
        \right)^2}}
    \end{equation}
    Normalization:
    \begin{equation}\nonumber
        \mathbf{v}[t] = \frac{\mathbf{v}[t] - \overline{x}}{s}, \qquad \forall 
        t \in \{1,\ldots,T\}.
    \end{equation}
\end{frame}

\begin{frame}{Multivariate Normalization I}
    \label{s-m-norm}
    Mean vector:
    \begin{equation}\nonumber
        E(\mathbf{V}[t]) = \vec E =
        \begin{bmatrix}
            \text{mean}(V_1[t]) \\
            \vdots \\
            \text{mean}(V_n[t])
        \end{bmatrix}
        =
        \begin{bmatrix}
            \frac{1}{T} \sum_{t=1}^T{V_1[t]} \\
            \vdots \\
            \frac{1}{T} \sum_{t=1}^T{V_n[t]} \\
        \end{bmatrix}.
    \end{equation}
    Covariance matrix:
    \begin{equation}\nonumber
        \text{Var}(\mathbf{V}[t]) =
        \Sigma_{n\times n}
        =
        \begin{bmatrix} 
            \text{cov}(V_1, V_1) & \dots  & \text{cov}(V_1, V_n) \\
            \vdots               & \ddots & \vdots \\
            \text{cov}(V_n, V_1) & \dots  & \text{cov}(V_n, V_n) \\ 
        \end{bmatrix}.
    \end{equation}
\end{frame}

\begin{frame}{Multivariate Normalization II}
    Covariance Function:
    \begin{equation}\nonumber
        \text{cov}(V_i[t], V_j[t]) = E\Big( \big[V_i[t] - E(V_i[t])\big] \cdot
            \big[V_j[t] - E(V_j[t])\big] \Big)
    \end{equation}
    Normalization following \cite{anacleto2020}
    \begin{equation}\nonumber
        \mathbf{V}[t] = \left(\Sigma_{n\times n}\right)^{-1/2} 
        \left(\mathbf{V}[t] - \vec E\right).
    \end{equation}
\end{frame}

\begin{frame}{Piecewise Aggregate Approximation}
    \label{s-paa}
    Following \cite{lin2003}, PAA is calculated as
    \begin{equation}\nonumber
        \overline{\mathbf{v}}[t] = \frac{w}{T} \sum_{j
        = \frac{n}{w}(t-1)+1}^{\frac{n}{w}t}{\mathbf{v}[t]},\qquad \forall t \in
        \{1,\ldots,w\}.
    \end{equation}
    Performs smoothing and dimension reduction.
\end{frame}

\begin{frame}{SAX Discretization I}
    \label{s-sax-disc}
    Idea: create $a$ equiprobable symbols for discretization.\\
    Action: split $\mathcal{N}(0, 1)$ into $a$ segments, use the resulting
    breakpoints. Breakpoints are $B = \beta_1,\ldots,\beta_{a-1}$. The area 
    under the  normal curve $\mathcal{N}(0,1)$ (i.e. the probability) between 
    two consecutive segments $\beta_i$ and $\beta_{i+1}=1/a$.
\end{frame}

\begin{frame}{SAX Discretization II}
\begin{table}[H]                                                                
    \centering                                                                  
    \caption{Breakpoint values for numbers of breakpoints $a$ from 3 to 6.   
    Table contents are quoted from \cite{lin2003}.}                           
    \begin{tabular}{|>{\columncolor{tablegray}}c | r | r | r | r |}\hline       
        \rowcolor{tablegray}                                                    
        {\centering\diagbox{$\beta_i$}{$a$}}& \multicolumn{1}{c|}{3} &          
        \multicolumn{1}{c|}{4} & \multicolumn{1}{c|}{5}  &                      
        \multicolumn{1}{c|}{6}   \\\hline                                       
        $\beta_1$   & -0.43 & -0.67  & -0.84 & -0.97    \\\hline                
        $\beta_2$   &  0.43 &   0  & -0.25 & -0.43    \\\hline                  
        $\beta_3$   & \multicolumn{1}{c|}{---} &  0.67  &  0.25 &   0    \\\hline
        $\beta_4$   & \multicolumn{1}{c|}{---} &\multicolumn{1}{c|}{---}&       
        0.84 &  0.43    \\\hline                                                
        $\beta_5$   & \multicolumn{1}{c|}{---} &\multicolumn{1}{c|}{---}&       
        \multicolumn{1}{c|}{---}&  0.97    \\\hline                             
    \end{tabular}                                                               
\end{table}
\end{frame}

\begin{frame}{SAX Discretization III}
    \centering
    \includegraphics[height=0.85\textheight]{msax}
\end{frame}

\begin{frame}{SAX/MSAX Distance Measure I}
    \label{s-mindist}
    SAX distance \cite{lin2003}:
    \begin{equation}\nonumber
MINDIST\left(\widehat{\mathbf{u}}[t], \widehat{\mathbf{v}}[t] \right) \equiv
    \sqrt{\frac{T}{w}}
    \sqrt{\sum_{t=1}^{w}{\left(dist(\widehat{\mathbf{u}}[t],
    \widehat{\mathbf{v}}[t]) \right)^2}}.
    \end{equation}
    MSAX distance \cite{anacleto2020}:
    \begin{equation}\nonumber
    MINDIST\_MSAX
        \left(
            \widehat{\mathbf{U}}[t], \widehat{\mathbf{V}}[t]
        \right)
    \equiv
    \sqrt{\frac{T}{w}}
    \sqrt{
        \sum_{t=1}^{w}{
            \left(
                \sum_{i=1}^{n}{
                    \left(
                        dist(\widehat{U_i}[t],\widehat{V_i}[t])
                    \right)^2
                }
            \right)
        }
    }.
\end{equation}
\end{frame}

\begin{frame}{SAX/MSAX Distance Measure II}
    $dist$ function \cite{lin2003}, is difference between breakpoints:
    \begin{equation}\nonumber
        \text{cell}_{r,c} =
        \left\{ \,
            \begin{aligned}
                0,
                    &\quad \text{if}\,\,\, |r-c| \le 1 \\
                \beta_{\text{max}(r,c)-1} - \beta_{\text{min}(r,c)},
                    &\quad \text{otherwise}
            \end{aligned}
        \right.
    \end{equation}
\end{frame}

\begin{frame}{SAX/MSAX Distance Measure III}
\begin{table}[H]                                                                
    \centering                                                                  
    \caption{A table for the $dist$ function for $a=5$. Each cell               
    displays the distance between the symbols denoting its row and column.}
    \begin{tabular}{|>{\columncolor{tablegray}}c | r | r | r | r | r |}\hline   
        \rowcolor{tablegray} & \multicolumn{1}{c|}{a} & \multicolumn{1}{c|}{b}  
            & \multicolumn{1}{c|}{c} & \multicolumn{1}{c|}{d} &                 
            \multicolumn{1}{c|}{e} \\\hline                                     
        a & 0       & 0        &0.59     &1.09     &1.68  \\\hline              
        b & 0       & 0        &0        &0.51     &1.09  \\\hline              
        c & 0.59    & 0        &0        &0        &0.59  \\\hline              
        d & 1.09    & 0.51     &0        &0        &0     \\\hline              
        e & 1.68    & 1.09     &0.59     &0        &0     \\\hline              
    \end{tabular}                                                               
\end{table}
\end{frame}

\begin{frame}{SAX/MSAX Distance Measure IV}
    \label{s-lb}
    Lower-bounding: infimum from set theory: largest value in set $S$ smaller 
    than all elements of a set $V \in S$.\\
    For SAX, MSAX: MINDIST is smaller than Euclidean Distance (``true"
    distance), thus it is representative
\end{frame}

\begin{frame}{HOT SAX/HOT MSAX Heuristic I}
    \label{s-hs}
    \begin{itemize}
        \item two parameters: $m$ and $k$
        \item two assumptions:
            \begin{itemize}[label=--]
                \item time series discords are rare
                \item segments similar to discords may also be discords
            \end{itemize}
        \item speed up discord discovery:
            \begin{itemize}[label=--]
                \item consider rarest segments first
                \item consider similar segments together
            \end{itemize}
    \end{itemize}
\end{frame}

\begin{frame}{HOT SAX/HOT MSAX Heuristic II}
    \centering                                                                  
    \includegraphics[height=0.85\textheight]{fig4}
\end{frame}

\begin{frame}{Statistical Analysis I}
    \label{s-stats}
\begin{table}[t]
    \centering
    \caption{Contingency table showing the relationship between detected
    discords and actual annotated values.}
    \begin{tabular}{|>{\columncolor{tablegray}}c | c | c |}\hline
        \rowcolor{tablegray}
        {\centering\diagbox{Actual}{Assigned}}& Discord Detected & Non-Discord
        Detected \\\hline
        Is Discord      & True Positive     & False Negative \\\hline
        Is Non-Discord  & False Positive    & True Negative  \\\hline
    \end{tabular}
\end{table}
\end{frame}

\begin{frame}{Statistical Analysis II}
\begin{equation}\nonumber                                                       
    \text{Recall} = \frac{\text{True Positive}}{\text{True Positive}            
    + \text{False Negative}},                                                   
\end{equation}                                                                  

\begin{equation}\nonumber                                                       
    \text{Accuracy} = \frac{\text{True Positive} + \text{True Negative}}        
        {\text{True Positive} + \text{True Negative} + \text{False Positive} +  
        \text{False Negative}},                                                 
\end{equation}                                                                  

\begin{equation}\nonumber                                                       
    \text{Precision} = \frac{\text{True Positive}}{\text{True Positive}         
    + \text{False Positive}}.                                                   
\end{equation}
\end{frame}

\begin{frame}{Finding Best Parameters -- Dataset 1}
    \label{s-params}
    Dataset 1 failed, thus dataset 2 was made
    \footnotesize
\begin{table}[H]
    \centering
    \caption{Table of the methods used for dataset 1, the parameters of each
    method, the rationale behind the parameter choice, and the values the
    parameter takes are shown.}
    \label{09:tab:ds1-param}
    \begin{tabular}{| c | c | c | c |}\hline
        \rowcolor{tablegray}
        Method & Parameter & Rationale & Values \\\hline
        \multirow{2}{*}{SAX/MSAX} & $w$ & arbitrary factors of 360 &
            2,3,4,5,12,20,30,40,60 \\\cline{2-4}
        & $a$ & arbitrary, 2$\le a \le$25 & 4,5,6,7,8,9,10,12,14,17,20
        \\\hline
        \multirow{4}{*}{HOT SAX/MSAX} & \multirow{2}{*}{$k$} & \multirow{2}{*}
        {arbitrary} & -1,25,50,100, \\
            & & & 150,200,300,500 \\\cline{2-4}
            & \multirow{2}{*}{$m$} & arbitrary factors of 360
            & \multirow{2}{*}{2,3,4,5,12,20,30,40,60} \\
            & & and of $w$ &\\\hline
    \end{tabular}
\end{table}
\end{frame}

\begin{frame}{Finding Best Parameters -- Dataset 1}
\begin{table}[H]
    \centering
    \caption{Results of the analysis of dataset 1. The total number of
    parameter sets and the number and proportion of
    parameter sets in dataset 1 that fulfill the analysis conditions are
    presented for each method.}
    \label{09:tab:ds1-results}
    \begin{NiceTabular}{| c | c | r | r |}\hline
        \cellcolor{tablegray} & \cellcolor{tablegray}
            &\multicolumn{2}{c|}{\cellcolor{tablegray}Sets Satisfying Analysis
            Conditions}
            \\\hhline{|>{\arrayrulecolor{tablegray}}->
                {\arrayrulecolor{tablegray}}->{\arrayrulecolor{black}}->
                {\arrayrulecolor{black}}|-|}
        \multirow{-2}{*}{\cellcolor{tablegray}Method} & \multirow{-2}{*}
            {\cellcolor{tablegray}Total Sets} & \multicolumn{1}{c|}
            {\cellcolor{tablegray}recall $\ge$95\%} & \multicolumn{1}{c|}
            {\cellcolor{tablegray}recall $\ge$95\% and $m \neq w$} \\\hline
        S-SAX   & \multirow{3}{*}{2,640} & 3  (0.1\%)   & 0 (0\%)
        \\\hhline{|>{\arrayrulecolor{black}}->{\arrayrulecolor{white}}->
            {\arrayrulecolor{black}}->{\arrayrulecolor{black}}|-|}
        D-SAX   &  & 13 (0.5\%)   & 0 (0\%)
        \\\hhline{|>{\arrayrulecolor{black}}->{\arrayrulecolor{white}}->
            {\arrayrulecolor{black}}->{\arrayrulecolor{black}}|-|}
        MSAX    &  & 23 (0.9\%)   & 3 (0.1\%) \\\hline
    \end{NiceTabular}
\end{table}
\end{frame}

\begin{frame}{Finding Best Parameters -- Dataset 2}
    \footnotesize
\begin{table}[H]                                                                
    \centering                                                                  
    \caption{Table of the methods used for dataset 2.}
    \begin{tabular}{| c | c | c | c |}\hline                                    
        \rowcolor{tablegray} Method & Parameter & Rationale & Values \\\hline   
        \multirow{5}{*}{SAX/MSAX} & \multirow{3}{*}{$w$} &                      
        \multirow{3}{*}{factors of 360}  & 2,3,4,5,6,8,9,10,12,15,  \\          
        & & & 18,20,24,30,36,40,45, \\                                          
          & & & 60,72,90,120,180,360 \\\cline{2-4}                              
           & \multirow{2}{*}{$a$} & 2$\le a \le$25, & \multirow{2}{*}           
           {$\overline{2,\ldots,25}$} \\                                        
          & & length of alphabet & \\\hline                                     
        \multirow{5}{*}{HOT SAX/MSAX}                                           
          & \multirow{2}{*}{$k$} & \multirow{2}{*}{arbitrary} &                 
            -1,25,50,75,100,            \\                                      
          & & & 150,175,200,300  \\\cline{2-4}                                  
           & \multirow{3}{*}{$m$}    & \multirow{3}{*}{same as $w$} &           
            2,3,4,5,6,8,9,10,12,15,  \\                                         
        & & & 18,20,24,30,36,40,45, \\                                          
          & & & 60,72,90,120,180,360 \\\hline                                   
    \end{tabular}                                                               
    \end{table}
\end{frame}

\begin{frame}{Finding Best Parameters -- S-SAX}
    \scriptsize
\begin{table}[H]                                                                
    \centering                                                                  
    \caption{Ranking of top 10 most precise S-SAX parameter combinations.}
    \begin{tabular}{|>{\columncolor{tablegray}} c | r | r | r | r | r | r | r |}
        \hline                                                                  
        \rowcolor{tablegray}                                                    
            \diagbox{Rank}{Properties} & $k$ & $w$ & $m$ & $a$ & Recall (\%)&   
            Accuracy (\%)& Precision (\%)\\\hline                               
        1& -1 &  40 &  40   &  19& 95.21& 37.46&35.94  \\\hline                 
        2& -1 &  24 &  24   &  24& 95.19& 37.57&35.93  \\\hline                 
        3& -1 &  30 &  30   &  22& 95.37& 37.46&35.93  \\\hline                 
        4& -1 &  36 &  36   &  20& 95.09& 37.41&35.92  \\\hline                 
        5& -1 &  45 &  45   &  18& 95.24& 37.33&35.89  \\\hline                 
        6& -1 &  24 &  24   &  25& 95.74& 37.32&35.88  \\\hline                 
        7& -1 &  72 &  72   &  15& 95.02& 36.86&35.87  \\\hline                 
        \cellcolor{gray}\textbf{8}& -1 &  36 &  36   &  21& 95.92& 37.06&35.86  \\\hline                 
        9& -1 &  30 &  30   &  23& 95.72& 37.17&35.86  \\\hline                 
        10& -1 & 120 & 120  &  13& 95.13& 36.51&35.85  \\\hline                 
    \end{tabular}                                                               
\end{table} 
\end{frame}

\begin{frame}[plain]
    \centering
    \includegraphics[width=\textwidth]{tssax_boxplot}
\end{frame}

\begin{frame}{Finding Best Parameters -- D-SAX}
    \scriptsize
\begin{table}[H]                                                                
    \centering                                                                  
    \caption{Ranking of top 10 most precise D-SAX parameter combinations.}
    \begin{tabular}{|>{\columncolor{tablegray}} c | r | r | r | r | r | r | r |}
        \hline
        \rowcolor{tablegray}                                                    
            \diagbox{Rank}{Properties} & $k$ & $w$ & $m$ & $a$ & Recall (\%)&   
            Accuracy (\%)& Precision (\%)\\\hline                               
        1 & -1 & 12 & 12 & 22 & 95.28& 41.91& 36.56\\\hline                     
        2 & -1 & 10 & 10 & 25 & 95.18& 41.99& 36.53\\\hline                     
        3 & -1 & 15 & 15 & 19 & 95.14& 41.70& 36.46\\\hline                     
        4 & -1 & 12 & 12 & 23 & 95.18& 41.00& 36.34\\\hline                     
        5 & -1 & 40 & 40 & 12 & 95.08& 39.64& 36.29\\\hline                     
        6 & -1 & 15 & 15 & 20 & 96.11& 40.31& 36.27\\\hline                     
        \cellcolor{gray}\textbf{7} & -1 & 12 & 12 & 24 & 97.04& 40.16& 36.26\\\hline                     
        8 & -1 & 36 & 36 & 13 & 95.80& 38.93& 36.20\\\hline                     
        9 & -1 & 20 & 20 & 17 & 95.87& 39.82& 36.19\\\hline                     
       10 & -1 & 30 & 30 & 14 & 96.09& 39.32& 36.18\\\hline                     
    \end{tabular}                                                               
\end{table}
\end{frame}

\begin{frame}[plain]
    \centering
    \includegraphics[width=\textwidth]{tdsax_boxplot}
\end{frame}

\begin{frame}{Finding Best Parameters -- MSAX}
    \scriptsize
\begin{table}[H]                                                                
    \centering                                                                  
    \caption{Ranking of top 10 most precise MSAX parameter combinations.}
    \begin{tabular}{|>{\columncolor{tablegray}} c | r | r | r | r | r | r | r |}
        \hline
        \rowcolor{tablegray}
            \diagbox{Rank}{Properties} & $k$ & $w$ & $m$ & $a$ & Recall (\%)&
            Accuracy (\%)& Precision (\%)\\\hline
            1 & -1 &  6 &  6 & 24 & 95.37& 40.68& 36.24\\\hline
            2 & -1 & 12 & 12 & 16 & 95.10& 39.85& 36.24\\\hline
            3 & -1 &  9 &  9 & 19 & 95.20& 39.70& 36.13\\\hline
            4 & -1 & 10 & 10 & 18 & 95.89& 39.45& 36.12\\\hline
            5 & -1 &  8 &  8 & 21 & 96.01& 39.53& 36.12\\\hline
            6 & -1 &  6 &  6 & 25 & 96.02& 39.94& 36.12\\\hline
            7 & -1 & 36 & 36 & 10 & 95.16& 38.47& 36.08\\\hline
            \cellcolor{gray}\textbf{8} & -1 & 12 & 12 & 17 & 96.51& 38.89& 36.06\\\hline
            9 & -1 & 30 & 30 & 11 & 95.49& 38.26& 36.03\\\hline
           10 & -1 & 72 & 72 &  8 & 95.70& 37.74& 36.03\\\hline
    \end{tabular}
\end{table}
\end{frame}

\begin{frame}[plain]
    \centering
    \includegraphics[width=\textwidth]{tmsax_boxplot}
\end{frame}

\begin{frame}{Finding Best Parameters -- Comparison}
    Biserial Correlation Analysis was performed with the goal of identifying the
    influence of the methods. It was found that:\\
    S-SAX vs D-SAX: 0.06\\
    S-SAX vs MSAX: 0.033\\
    D-SAX vs MSAX: -0.04
\end{frame}

\end{document}
