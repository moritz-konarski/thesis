\documentclass[../00_main.tex]{subfiles}

\begin{document}

\section{\textcite{collifranzone1985}}

\fullcite{collifranzone1985}

\subsubsection*{Summary}

\begin{itemize}
    \item Cauchy problem for elliptic operator that's strongly ill posed
    \item solution is regularized
    \item electrocardiography potential problem: body surface potential from
        epicardial potential, also inverse problem
    \item epicardial data can be estimated using surface potential data
    \item estimates are good because they contain information not really
        available purely on the surface
    \item high errors are common with this procedure because: (1)
        simplification of body inhomogeneity, (2) reconstruction of data on
        static heart, (3) constant (over heart beat) smoothing parameter
    \item computed epicardial potential maps (CEPMs) still not clear
    \item not having the heart centered could improve accuracy for one side of
        the heart
    \item new experiment with a dog heart in a child-sized torso (roughly
        appropriate location)
    \item detailed analysis of potential estimates
    \item only errors should be noise in measurements, misaligned electrodes
    \item finite element matrix links surface to epicardial potential, fine 3D
        surface grid
    \item optimized for CEPM accuracy
    \item efficiency of choosing a time-dependent smoothing parameter
    \item applied inverse procedure to heart beat to reconstruct ECG
    \item for experiment description read the paper
    \item tank surface is $\Gamma_1$, epicardial frame surface is $\Gamma_0$,
        and $\Omega$ the volume bounded by $\Gamma_0$, $\Gamma_1$
    \item we have electrodes on the gammas and can approximate omega
    \item the electric cardiac potential $V(x,t)$ is the solution to
        \begin{equation}\nonumber
            \begin{gathered}
                \Delta V(x, t) = V_{xx} + V_{yy} + V_{zz} = 0 \qquad
                \text{in}\, \Omega,\\
                V(x,t) = u(x,t) \qquad \text{on}\, \Gamma_0,\\
                \frac{\partial V(x,t)}{\partial n} = 0 \qquad \text{on}\,
                \Gamma_1
            \end{gathered}
        \end{equation}
    \item $u(x,t)$ is the epicardial potential distribution at $t$, $n$ is the
        normal vector to $\Gamma_1$ (from the insulating layer around the whole
        thing, nothing gets out)
    \item trace of $V$ on $\Gamma_1$ is the thorax potential distribution,
        denoted by $z(x,t)$
    \item we have a linear operator $A$ transforming $u$ to $z$
    \item direct problem of electrocardiography is solving the above system of
        equations
    \item problem is discretized using finite-element methods 
\end{itemize}

\end{document}
