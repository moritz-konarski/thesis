\documentclass[../01_main.tex]{subfiles}

\begin{document}

\section{Results}

\TODO{Sections
    \\- use confusion matrices for what is is vs what was predicted [p. 44
    anacleto2019]
    \\- compare all the parameters and their influence
    \\- 
}

\subsection{First Run}

\begin{itemize}
    \item parameters for this run
    \item why could I not let it continue
    \item what did this run indicate -> what did I change and modify for the
        next run
    \item keep in mind that the lower recall can be caused by the way I do the
        ECG checking, and that I did not want to assign data to segments that
        did not have it before out of fear that I would invent results.
\end{itemize}

\subsection{Second Run}

\subsection{SAX}

influence and significance of all the major parameters:

\begin{itemize}
    \item k
    \item paa count
    \item subsequence count
    \item alphabet size
    \item which ones seem to be the best
\end{itemize}

\subsection{MSAX}

influence and significance of all the major parameters:

\begin{itemize}
    \item k
    \item paa count
    \item subsequence count
    \item alphabet size
    \item which ones seem to be the best
\end{itemize}

\subsection{MSAX vs SAX}

Comparing SAX to MSAX is done using the recall value defined in
\TODO{reference}. Investigating the correlation between the methods
(represented by a 1 for SAX and a 0 for MSAX), yields the correlation
coefficient of -0.25. This coefficient indicates that for all investigated
parameter combinations, the use of the MSAX method is weakly correlated with an
increase in recall. When a specific set of parameters is selected and the
correlation analysis is repeated, the correlation coeficient is -0.73,
indicating a strong correlation. Here $k=-1$ and paa\_count = 12.

\begin{itemize}
    \item just the results that are gained directly from the data
    \item put results in graphs and tables to make them referencable
\end{itemize}

\end{document}
