\documentclass[../01_main.tex]{subfiles}

\begin{document}

\section{Results}

\subsection{Implementation}

This subsection is concerned with the implementation of the methods discussed
in the previous section. First, the ECG data and its preparation will be
discussed, followed by notes on the implementation of the SAX, MSAX, and HOT
SAX methods. Lastly, the process used to analyze the results is discussed.
All code used in the implementation of these methods is available upon request
via email at \Verb+konarski_m@auca.kg+.

\subsection{ECG Analysis Program}

\subsubsection{ECG Preprocessing}

The ECG data used in this project can be downloaded using the PhysioNet 
website at the url
\url{https://www.physionet.org/content/mitdb/1.0.0/}, or, alternatively, using
the PhysioNet-developed WFDB applications package. This package provides
command line applications to work with PhysioNet data. For each of the
individually numbered ECG records, 4 files exist. The
\Verb_.hea_ files contain metadata on the ECG record,
including anonymized patient information and the lead names. The
\Verb_.dat_ files contain the actual ECG recording and
the other two files contain additional information, including the annotations.
Once the ECG recording has been downloaded, the
\Verb_rdsamp_ command is used to convert the binary ECG
recording files into a more user-friendly comma separated value (CSV) file.
The \Verb_rdann_ command is then used to create a CSV
file containing the annotations for each of the ECG records. Finally, the ECG
recording data and the annotations can be merged into a single file by using
the time stamps contained in both files. This yields full ECG recordings with
added beat annotations in one file. These files are the basis of all further
methods and analysis performed in this research. The author created a script in
the Julia programming language that performs this process. Filtering of the ECG
data is not performed. The rationale behind this is twofold. Firstly, the
combination of PAA and discretization in SAX and MSAX has a smooting effect
that exhibits some of the same properties as filtering. Additionally, filtering
of ECGs adds many more parameters that can be modified to improve the
performance of the methods, which is not desirable for this research as the
methods should depend on the least possible number of parameters for
simplicity. As additional support for this approach, \mycite{zhang2019} can be
considered, which successfully uses SAX in their ECG analysis without
mentioning any filtering performed on the ECG data.

\subsubsection{SAX, MSAX, HOT SAX, and HOT MSAX}

For the analysis to start, a set of parameters $k$, $w$, $m$, and $a$ needs to
be set. Then, the next step is to load a CSV file containing the ECG data and 
annotations into the program. This files was generated in the preprocessing
step. Once the ECG file is loaded, it is transformed into a data frame.
A data frame is a type of data structure that can hold heterogeneous data
types, e.g. text and numbers. This step adds important information to the ECG
data. The ECG data frame contains the parameters itemized above to enable
reproduction and analysis of the results, an index range for each PAA segment
so it can be located in the raw ECG, the beat annotations for each PAA segment,
and empty data fields for the results of the analysis with HOT SAX and HOT MSAX. 
The next
step is the application of the SAX and MSAX representations. The transformation
of the raw time series data to the symbolic representations is performed in the
same order as discussed in the methods section. Thanks to the Julia
programming language's ecosystem of libraries, the formulas and functions can 
be easily translated into code. SAX is applied to each of the ECG leads 
individually, while MSAX is applied as designed to both at once. HOT SAX and 
HOT MSAX are the next step. For HOT MSAX,
the process is performed using the MSAX representation and distance
measure. The method returns a list of distances as well as a list of indices
that indicate which PAA segment has which distance. Depending on the parameter
$k$, only the top $k$ of these discords are returned. These results are then
added to the respective PAA segments in the ECG data frame, adding both the
MSAX distance of the segment as well as a binary indicator of whether or not
the segment was detected as a discord. For SAX the process slightly different.
Because SAX is a univariate representation, it cannot be directly applied to
a bivariate ECG. Thus, HOT SAX is applied to each lead of the ECG separately. 
Each set of results is, like for HOT MSAX, a list of indices of PAA segments 
and a list of their distances.
Each set of results is also added to the ECG data frame. This time the
detection indicator is quaternary, it represents no detection, detection on the
first lead, detection on the second lead, or detection on both leads. After
both of these processes are completed, the ECG data frame is written to a CSV
file for further analysis. This process can be repeated thousands of times to
create data of different values for the parameters to determine optimal values
and their influence.

\subsubsection{Computerized Statistical Analysis}

The analysis of the methods is performed using the data whose generation was
discussed above. The first step in the analysis 
is the processing of the data generated using the Julia program. This
consisted of calculating the True Positive, True Negative, False Negative, and
False Positive values for each parameter combination and each method.
Here, the HOT SAX and HOT MSAX methods can be segmented into three datasets.
The dataset generated using HOT MSAX stays the same is from now on referred to
simply as MSAX. The data generated with HOT SAX can be divided into two
distinct data sets. The first one treats SAX like a univariate representation
and is thus called Single SAX (S-SAX). For S-SAX, the discords which were
discovered by HOT SAX on one lead are treated separately from the discords
discovered on the second lead. This means that a segment is counted as
a discord if it was discovered by one method or by both methods. Effectively
this generates two sets of results which can be concatenated. S-SAX represents
the performance of applying SAX to just one lead of the ECG. The second option
is a middle ground between S-SAX and MSAX. For this dataset, a discord was
counted if it was discovered by either the first or the second lead or both. As
a result, SAX is, in a way, applied to a multivariate time series. Because SAX
is applied to both ECG leads, this dataset is called Dual SAX (D-SAX).\par

To calculate the values shown in \tabref{08:tab:bin-class}, the meaning of a 
true discord and a true non-discord in the ECG needs to be defined.
A segment is considered a ``non-discord" if its annotation consisted of an
``N" or nothing ``". The former is obvious; the decision to consider no
annotation (``") a non-discord was made because for certain segments of the
ECGs, no annotations were available. This can happen if, for example, the 
subsequence length for HOT SAX is much smaller than one heartbeat. In that
situation, one heartbeat might be represented by 5 or more subsegments. The
heartbeat annotation, given for a specific point in time, will only fall into
one of the 5 segments and can thus only be counted for that one segment. The
same is true for an annotation showing a discord. This method of analysis puts
HOT SAX and HOT MSAX at a disadvantage because a discord located in one 
subsegment might
influence its neighboring segments and thus lead to their detection. This
detection might be an actual discord being detected, but counting it as one
would incorrectly inflate the True Positive rate by assuming something about
the data that it itself does not support without some inference. Thus the
decision was made to accept lower True Positive values than may be
accurate.\par

Any annotation that
was not normal as just defined  was considered a discord. This includes the 
medical annotations for arrhythmia but also the annotations for changes in signal
quality or noise. This is done because, as mentioned, HOT SAX and HOT MSAX are 
not meant to classify heartbeats by medical significance, but by how different 
they are from other heartbeats. A very noisy normal heartbeat will be detected 
the same as an arrhythmic beat. The classification of the detected discords 
into medically normal and abnormal heartbeats is left to more sophisticated 
analysis methods or human experts. The purpose of the HOT SAX and HOT MSAX 
methods is merely to reduce the number of ECG segments that need to be analyzed by pre-selecting the beats
likely to contain useful information.\par

After calculating the True Positive, 
True Negative, False Negative, and False Positive value for each parameter
combination, they were collected in a data frame also containing information on 
the parameters that lead to them. These data frames are then saved as CSV files
for further analysis. The contingency values were analyzed for the three
datasets mentioned above, S-SAX, D-SAX, and MSAX. The last
step in the analysis was the calculation of the average recall, precision, and
accuracy across all 48 ECGs for each parameter set.

\subsection{Limitations of the Implementation}

The disussion of the definition of normal and abnormal heartbeats above 
explains why empty annotations have
to be counted as normal beats and while the author believes that this is 
necessary, it does negatively influence the results. Another limitation is that 
the implementation of the SAX and MSAX
representations only allows the use of PAA segment numbers that evenly divide
the sampling frequency of the ECG database. This was done so that the whole
ECG, being an even multiple of the sampling frequency itself, can be evenly
divided into PAA segments. This decision prohibits certain numbers of PAA
segments as there may be numbers that do not evenly divide into the sampling
frequency but that do evenly divide the number of raw data points in the ECG.
A further simplification step in the same vein is the restriction of
subsequence values to numbers that evenly divide the number of PAA segments
$w$. This was also done to simplify the process and to guarantee that the whole
ECG would be evenly divisible into subsequences.

\subsection{Data Analysis}

In this section, the results of the data analysis will be presented. For both
the first and second datasets used in the research, the parameter selection and
a summary of the results will be presented. For dataset 2, the analysis will be
completed in accordance with the methods laid out in section 
\ref{08:stat-analysis}.

\subsubsection{Dataset 1}

The first dataset is based on parameters that were
arbitrarily chosen. This was done because the behavior of the methods was not
yet known. While the choice was arbitrary, it was attempted to spread the
values out in a reasonable way. For each parameter, \tabref{09:tab:ds1-param} 
shows the method that it influences, the values that were chosen for it, and
the rationale behind choosing those values. 
For SAX and MSAX, $w$ the number of PAA segments in one second which is 
equivalent to dividing 360 data points by $w$. Thus, $w$ has to be a factor of 
360. These factors were chosen arbitrarily and kept small because it was not
known how time intensive the computations would be. The second parameter is 
$a$, the alphabet size which determines the number of different symbols used in 
the SAX and MSAX discretization processes. Because the alphabet size is 
constrained by the size of the English alphabet, numbers were chosen 
arbitrarily in that range.
For HOT SAX and HOT MSAX, the parameter $k$ is number of discords they return.
Giving it the value $-1$ indicates that all available discords should be
returned, regardless of how many there are. The values for $k$ were chosen
arbitrarily. This parameter does not affect the performance of the method, it
just determines how many of the results are considered. Parameter $m$ is chosen 
after parameter $w$ for SAX and MSAX and represents the number of PAA segments 
that are grouped together to form a HOT SAX or HOT MSAX subsequence. This 
parameter must evenly divide $w$. 
\begin{table}[H]
    \centering
    \caption{Table of the methods used for dataset 1. the parameters of each
    method, the rationale behind the parameter choice, and the values the
    parameter takes are shown.}
    \label{09:tab:ds1-param}
    \vspace{0.5em}
    \begin{tabular}{| c | c | c | c |}\hline
        \rowcolor{tablegray}
        Method & Parameter & Rationale & Values \\\hline 
        \multirow{2}{*}{SAX/MSAX} & $w$ & arbitrary factors of 360 & 
            $2,3,4,5,12,20,30,40,60$ \\\cline{2-4}
        & $a$ & arbitrary, $2 \le a \le 25$ & $4,5,6,7,8,9,10,12,14,17,20$  
        \\\hline
        \multirow{4}{*}{HOT SAX/MSAX} & \multirow{2}{*}{$k$} & \multirow{2}{*}
        {arbitrary} & $-1,25,50,100,$ \\
            & & & $150,200,300,500$ \\\cline{2-4} 
            & \multirow{2}{*}{$m$} & arbitrary factors of 360
            & \multirow{2}{*}{$2,3,4,5,12,20,30,40,60$} \\
            & & and of $w$ &\\\hline
    \end{tabular}
\end{table}
Using these parameters and the programs created as part of this research, the
48 ECGs of the MIT-BIH database were analyzed using HOT SAX and HOT MSAX. The 
2,640 unique sets of parameters resulting from \tabref{09:tab:ds1-param}
applied to 48 ECGs creates a dataset with 126,720 files. These
files were analyzed as stated in section \ref{08:stat-analysis}.
The mean values of the statistical measures for each set of unique parameters 
were calculated. Then, the threshold of recall $\ge 95\%$ was applied to the
summarized data to select the parameter combinations that yield acceptable
results. Upon further analysis, it was noted that most of the parameter
combinations that achieved recall $\ge 95\%$ had $m=w$. To investigate this, the
parameter combinations with $m \neq w$ and recall $\ge 95\%$ were extracted.
\tabref{09:tab:ds1-results} shows the results of this analysis.
\begin{table}[H]
    \centering
    \caption{Results of the analysis of dataset 1. The total number of
    parameter sets and the number and proportion of
    parameter sets in dataset 1 that fulfill the conditions analysis are 
    presented for each method.}
    \label{09:tab:ds1-results}
    \vspace{0.5em}
    \begin{NiceTabular}{| c | c | r | r |}\hline
        \cellcolor{tablegray} & \cellcolor{tablegray}
            &\multicolumn{2}{c|}{\cellcolor{tablegray}Sets Satisfying Analysis 
            Conditions}
            \\\hhline{|>{\arrayrulecolor{tablegray}}->
                {\arrayrulecolor{tablegray}}->{\arrayrulecolor{black}}->
                {\arrayrulecolor{black}}|-|}
        \multirow{-2}{*}{\cellcolor{tablegray}Method} & \multirow{-2}{*}
            {\cellcolor{tablegray}Total Sets} & \multicolumn{1}{c|}
            {\cellcolor{tablegray}recall $\ge95\%$} & \multicolumn{1}{c|}
            {\cellcolor{tablegray}recall $\ge95\%$ and $m \neq w$} \\\hline
        S-SAX   & \multirow{3}{*}{$2,640$} & $3 $ $(0.1\%)$   & $0$ $(0\%)$ 
        \\\hhline{|>{\arrayrulecolor{black}}->{\arrayrulecolor{white}}->
            {\arrayrulecolor{black}}->{\arrayrulecolor{black}}|-|}
        D-SAX   &  & $13$ $(0.5\%)$   & $0$ $(0\%)$ 
        \\\hhline{|>{\arrayrulecolor{black}}->{\arrayrulecolor{white}}->
            {\arrayrulecolor{black}}->{\arrayrulecolor{black}}|-|}
        MSAX    &  & $23$ $(0.9\%)$   & $3$ $(0.1\%)$ \\\hline
    \end{NiceTabular}
\end{table}
As \tabref{09:tab:ds1-results} shows, less than $1\%$ of the parameter sets
have a recall of $\ge 95\%$, regardless of the method used. Additionally, only
3 ($0.1\%$) parameter sets for MSAX have the desired recall and use a value for
$m$ that is different from $w$.
These sets of results are too small to continue this analysis, as the goal is
to analyze a ``top 10" selection of values. Thus, a second
dataset needs to be computed. The data presented in \tabref{09:tab:ds1-results}
does show that using subsequence lengths $m$ for the HOT SAX and HOT MSAX 
methods that are not equal to the PAA segment count $w$ is not effective.
According to these findings, $m$ will be set equal to $w$ for the computation
of dataset 2.

\subsubsection{Dataset 2}

Based on the analysis of dataset 1, the parameter selection for dataset 2 is
optimized. As the value of $m$ is set equal to the value of $w$, the complexity
of the computation is reduced dramatically. This is caused by two things.
Firstly, the total number of parameter sets is decreased when factors $m$ of
$w$ are not considered--the total number of parameter sets that have to be
used it decreased. Secondly, dividing the number of PAA segments in 1 second
into multiple subsegments increases the number of subsequences that HOT SAX and
HOT MSAX need to work with and thus the complexity of the computation. By not
doing that, the increase in complexity is avoided. As a result of this
reduction in complexity, a larger set of the other parameter was considered. 
The parameters chosen for dataset 2 are shown in \tabref{09:tab:ds2-param}. For
the SAX and MSAX parameters, all possible values were considered. Parameter 
$w$ can be all factors of $360$. For the alphabet size $a$, all possible values 
were considered. The HOT SAX and HOT MSAX parameters were chosen as follows.
Parameter $k$ was again assigned arbitrary values that provide a decent
coverage for reasonable values. The value of $-1$ is again included to signify
the use of all available discords.
Parameter $m$ does not need to be chosen for this dataset, as it is always set
to the value of $w$. All parameters used in dataset 2 have the same meaning as 
in dataset 1, please refer to that section for their explanations. 
\begin{table}[H]
    \centering
    \caption{Table of the methods used for dataset 2. The parameters of each
    method, the rationale behind the parameter choice, and the values the
    parameter takes are shown.}
    \label{09:tab:ds2-param}
    \vspace{0.5em}
    \begin{tabular}{| c | c | c | c |}\hline
        \rowcolor{tablegray} Method & Parameter & Rationale & Values \\\hline
        \multirow{5}{*}{SAX/MSAX} & \multirow{3}{*}{$w$} & 
        \multirow{3}{*}{factors of 360}  & $2,3,4,5,6,8,9,10,12,15,$  \\
        & & & $18,20,24,30,36,40,45,$ \\
          & & & $60,72,90,120,180,360$ \\\cline{2-4}
           & \multirow{2}{*}{$a$} & $2 \le a \le 25,$ & \multirow{2}{*}
           {$\overline{2,\ldots,25}$} \\    
          & & length of alphabet & \\\hline
        \multirow{3}{*}{HOT SAX/MSAX}
          & \multirow{2}{*}{$k$} & \multirow{2}{*}{arbitrary} & 
            $-1,25,50,75,100,$            \\
          & & & $150,175,200,300$  \\\cline{2-4}
           & $m$    & same as $w$           & see $w$ \\\hline
    \end{tabular}
\end{table}
This table provides values that create 4,968 parameter combinations when used
with this research's programs to analyze the 48 ECGs of the MIT-BIH database.
As a result, dataset 2 contains 238,464 files of ECGs analyzed with HOT SAX and
HOT MSAX. Each of those combinations was applied to all 48 ECGs. 
As with dataset 1, these files were analyzed as stated in section 
\ref{08:stat-analysis}. The mean values of the statistical measures for each set 
of unique parameters were calculated and the threshold of recall $\ge 95\%$ 
applied. \tabref{09:tab:ds2-results} shows the results of that analysis.
\begin{table}[H]
    \centering
    \caption{Results of the analysis of dataset 2. The total number of
    parameter sets and the number and proportion of
    parameter sets fulfilling the conditions that recall be $\ge 95\%$.}
    \label{09:tab:ds2-results}
    \vspace{0.5em}
    \begin{NiceTabular}{| c | c | r |}\hline
        \rowcolor{tablegray}
        Method & Total Sets & \multicolumn{1}{c|}{Sets Satisfying recall
        $\ge95\%$} \\\hline
        S-SAX   & \multirow{3}{*}{$4,968$} & $99$ $(1.2\%)$    
        \\\hhline{|>{\arrayrulecolor{black}}->{\arrayrulecolor{white}}->
            {\arrayrulecolor{black}}|-|}
        D-SAX   &  & $192$ $(3.9\%)$    \\\hhline{|>{\arrayrulecolor{black}}->
        {\arrayrulecolor{white}}->{\arrayrulecolor{black}}|-|}
        MSAX    &  & $255$ $(5.1\%)$    \\\hline
    \end{NiceTabular}
\end{table}
As \tabref{09:tab:ds2-results} shows, of the 4,968 total parameter sets, 99 have
a recall of $\ge 95\%$ for S-SAX, 192 for D-SAX, and 255 for MSAX. These
dataset are large enough for the analysis to continue. As discussed in
\ref{08:stat-analysis}, the subsets of dataset 2 for
which the recall is $\ge 95\%$ will further be sorted by the precision, in
descending order. Then, the top 10 values of S-SAX, D-SAX, and MSAX will be
analyzed individually. 

\paragraph{Analysis of S-SAX}

The top 10 values of S-SAX were first pruned by a threshold of recall $\ge
95\%$ and then sorted descendingly by precision. \tabref{09:tab:ssax10}
provides an overview of the parameters of these top 10 values, as well as their
recall, accuracy, and precision.
\begin{table}[H]
    \centering
    \caption{Table presenting a ranking of the top 10 most precise S-SAX
    parameter combinations and their parameters $k$, $w$, $m$, and $a$. The
    recall and accuracy values are also shown.}
    \label{09:tab:ssax10}
    \vspace{0.5em}
    \begin{tabular}{|>{\columncolor{tablegray}} c | r | r | r | r | r | r | r |}
        \hline
        \rowcolor{tablegray} 
            \diagbox{Rank}{Properties} & $k$ & $w$ & $m$ & $a$ & Recall (\%)& 
            Accuracy (\%)& Precision (\%)\\\hline
        1& -1 &  40 &  40   &  19& 95.21& 37.46&35.94  \\\hline
        2& -1 &  24 &  24   &  24& 95.19& 37.57&35.93  \\\hline
        3& -1 &  30 &  30   &  22& 95.37& 37.46&35.93  \\\hline
        4& -1 &  36 &  36   &  20& 95.09& 37.41&35.92  \\\hline
        5& -1 &  45 &  45   &  18& 95.24& 37.33&35.89  \\\hline
        6& -1 &  24 &  24   &  25& 95.74& 37.32&35.88  \\\hline
        7& -1 &  72 &  72   &  15& 95.02& 36.86&35.87  \\\hline
        8& -1 &  36 &  36   &  21& 95.92& 37.06&35.86  \\\hline
        9& -1 &  30 &  30   &  23& 95.72& 37.17&35.86  \\\hline
        10& -1 & 120 & 120  &  13& 95.13& 36.51&35.85  \\\hline
    \end{tabular}
\end{table}
\tabref{09:tab:ssax10} shows that for the top 10 values by precision, the
recall values are all approximately 95\%, the accuracy is between 36\% and
38\%, and the precision is 36\%. It is notable that all $k$ values are -1. To 
be able to choose a best parameter
combination of these 10, their interquartile ranges and outliers for the recall
value will be compared with the help of a boxplot. This boxplot is based on the
set of 48 ECGs for each ranked method. This plot can be seen in 
\figref{09:fig:ssax_boxplot}.
\begin{figure}[H]
    \centering
    \includegraphics[width=\textwidth]{ssax_boxplot}
    \caption{Boxplot showing the recall for the top 10 S-SAX parameter sets. The
    full list of parameters can be found in \tabref{09:tab:ssax10}.}
    \label{09:fig:ssax_boxplot}
\end{figure}
\figref{09:fig:ssax_boxplot} illustrates two important things. Firstly, there
are only small differences between the recall values for any of the top 10
S-SAX parameter sets. Secondly, a clear pattern of 9 outliers is visible.

To select the optimal parameter based on \figref{09:fig:ssax_boxplot}, the
interquartile range (IQR) is calculated for each rank. The use of IQR versus
the mean is justified here because of the skewed data and the outliers. Ranks 
6, 8, and 9 have 
the lowest IQR, at 0.0358, 0.0354, and 0.378, respectively. For further
differentiation, the number of outliers is considered. Rank 6 has 13 outliers,
rank 8 11, and rank 9 has 13. Considering these factors, rank 8 is chosen as
the optimal parameter set for S-SAX. This set has the smallest IQR of the top 
10 and the
smallest number of outliers among those with small IQR values. With a $w$ value
of 36, rank 8 represents a 10 time dimension reduction compared to the raw
data.

\paragraph{Analysis of D-SAX}

For D-SAX, the top 10 values were also first pruned by a threshold of recall 
$\ge 95\%$ and then sorted descendingly by precision. \tabref{09:tab:dsax10}
provides an overview of the parameters of these top 10 values, as well as their
recall, accuracy, and precision.
\begin{table}[H]
    \centering
    \caption{Table presenting a ranking of the top 10 most precise D-SAX
    parameter combinations and their parameters $k$, $w$, $m$, and $a$. The
    recall and accuracy values are also shown.}
    \label{09:tab:dsax10}
    \vspace{0.5em}
    \begin{tabular}{|>{\columncolor{tablegray}} c | r | r | r | r | r | r | r |}\hline
        \rowcolor{tablegray} 
            \diagbox{Rank}{Properties} & $k$ & $w$ & $m$ & $a$ & Recall (\%)& 
            Accuracy (\%)& Precision (\%)\\\hline
        1 & -1 & 12 & 12 & 22 & 95.28& 41.91& 36.56\\\hline
        2 & -1 & 10 & 10 & 25 & 95.18& 41.99& 36.53\\\hline
        3 & -1 & 15 & 15 & 19 & 95.14& 41.70& 36.46\\\hline
        4 & -1 & 12 & 12 & 23 & 95.18& 41.00& 36.34\\\hline
        5 & -1 & 40 & 40 & 12 & 95.08& 39.64& 36.29\\\hline
        6 & -1 & 15 & 15 & 20 & 96.11& 40.31& 36.27\\\hline
        7 & -1 & 12 & 12 & 24 & 97.04& 40.16& 36.26\\\hline
        8 & -1 & 36 & 36 & 13 & 95.80& 38.93& 36.20\\\hline
        9 & -1 & 20 & 20 & 17 & 95.87& 39.82& 36.19\\\hline
       10 & -1 & 30 & 30 & 14 & 96.09& 39.32& 36.18\\\hline
    \end{tabular}
\end{table}
\tabref{09:tab:dsax10} shows that for the top 10 values by precision, like in
the previous section. The 
recall values are between 95\% and 97\%, the accuracy between 39\% and
42\%, and the precision is 36\%. Again, all $k=-1$. A boxplot comparing recall 
with rank is
created in order to choose a best parameter combination. This plot is shown in 
\figref{09:fig:dsax_boxplot}.
\begin{figure}[H]
    \centering
    \includegraphics[width=\textwidth]{dsax_boxplot}
    \caption{Boxplot showing the recall for the top 10 D-SAX parameter sets. The
    full list of parameters can be found in \tabref{09:tab:dsax10}.}
    \label{09:fig:dsax_boxplot}
\end{figure}
Visually, \figref{09:fig:dsax_boxplot} shows ranks 6, 7, and 9 to have the
smallest IQR. This is confirmed by calculating the IQR. Rank 6 has an IQR of
0.038, rank 7 has 0.03, and rank 9 has 0.04. Regarding the number of outliers,
rank 7 has 6, rank 6 has 7, and rank 9 has 8. Thus rank 7 is selected as the
optimal parameter set. It has both the smallest IQR and the lowest number of
outliers compared to other ranks with low IQRs. The dimension reduction ratio
of rank 7 is 30.

\paragraph{Analysis of MSAX}

Lastly, the top 10 parameter sets of MSAX are considered. They, too, were 
selected based on recall $\ge 95\%$ and sorted by precision. 
\tabref{09:tab:msax10} shows the parameter set and statistical measures 
associated with the top 10 MSAX parameter combinations.
\begin{table}[H]
    \centering
    \caption{Table presenting a ranking of the top 10 most precise MSAX
    parameter combinations and their parameters $k$, $w$, $m$, and $a$. The
    recall and accuracy values are also shown.}
    \label{09:tab:msax10}
    \vspace{0.5em}
    \begin{tabular}{|>{\columncolor{tablegray}} c | r | r | r | r | r | r | r |}\hline
        \rowcolor{tablegray} 
            \diagbox{Rank}{Properties} & $k$ & $w$ & $m$ & $a$ & Recall (\%)& 
            Accuracy (\%)& Precision (\%)\\\hline
            1 & -1 &  6 &  6 & 24 & 95.37& 40.68& 36.24\\\hline
            2 & -1 & 12 & 12 & 16 & 95.10& 39.85& 36.24\\\hline
            3 & -1 &  9 &  9 & 19 & 95.20& 39.70& 36.13\\\hline
            4 & -1 & 10 & 10 & 18 & 95.89& 39.45& 36.12\\\hline
            5 & -1 &  8 &  8 & 21 & 96.01& 39.53& 36.12\\\hline
            6 & -1 &  6 &  6 & 25 & 96.02& 39.94& 36.12\\\hline
            7 & -1 & 36 & 36 & 10 & 95.16& 38.47& 36.08\\\hline
            8 & -1 & 12 & 12 & 17 & 96.51& 38.89& 36.06\\\hline
            9 & -1 & 30 & 30 & 11 & 95.49& 38.26& 36.03\\\hline
           10 & -1 & 72 & 72 &  8 & 95.70& 37.74& 36.03\\\hline
    \end{tabular}
\end{table}
\tabref{09:tab:msax10} shows, for the third time, that there is little
difference between the statistical results of the top 10 best parameter sets. 
The recall values are all 95\% to 96\%, the accuracy is between 38\% and
41\%, and the precision is 36\%. A boxplot is constructed for these 10
parameter sets to explore the interquartile range and outliers of MSAX.
\figref{09:fig:ssax_boxplot} shows this boxplot.
\begin{figure}[H]
    \centering
    \includegraphics[width=\textwidth]{msax_boxplot}
    \caption{Boxplot showing the recall for the top 10 MSAX parameter sets. The
    full list of parameters can be found in \tabref{09:tab:msax10}.}
    \label{09:fig:msax_boxplot}
\end{figure}
In \figref{09:fig:msax_boxplot}, ranks 6, 8, and 10 visually stick out as 
having the lowest IQR. Calculating the IQR confirms that, yielding 0.038,
0.033, and 0.03 for ranks 6, 8, and 10 respectively. The number of outliers for
rank 6 is 7, for rank 8 it is 6, and for rank 10 it is 8. Because rank
8 has the second lowest IQR and the smallest number of outliers, it is selected
as the optimal parameter set for MSAX. Its $w$ value of 12 results in
a dimension reduction of 30 compared to the raw data.

\subsubsection{Comparison of Optimal Parameters}

After determining the optimal parameter sets for the three methods investigated
with the help of dataset 2, they will now be compared with each other. \tabref{09:tab:opt-param}
summarizes the parameter sets that are optimal for S-SAX, D-SAX, and MSAX. When
it comes to the method parameters, it is desirable for $w$ to be as small as
possible to take advantage of the dimension reduction properties of SAX and
MSAX. Accordingly, D-SAX and MSAX have the best $w$ parameters, as they are
smaller than $w$ of S-SAX. Considering parameter $k$, they all have $k=-1$. The
alphabet size $a$, while not influencing the dimension reduction, is relevant
for the complexity of the method--lower values of $a$ result in a simpler
representation. Accordingly, MSAX has the best value for $a$ because it is
smaller than the competing methods.
\begin{table}[H]
    \centering
    \caption{Table showing the optimal sets of parameters for S-SAX, D-SAX, and
    MSAX. Parameters $w$ and $m$ are combined because they are equal to each
    other. Best parameters are highlighted in bold.}
    \label{09:tab:opt-param}
    \vspace{0.5em}
    \begin{tabular}{|>{\columncolor{tablegray}} c | r | r | r |}\hline
        \rowcolor{tablegray} 
            \diagbox{Method}{Parameter} & $k$ & $w,m$  & $a$ \\\hline
        S-SAX   & -1    & 36             & 21               \\\hline
        D-SAX   & -1    & \textbf{12}    & 24               \\\hline
        MSAX    & -1    & \textbf{12}    & \textbf{17}      \\\hline
    \end{tabular}
\end{table}
To further compare these sets of parameters, a boxplot of their recall values
is created. \figref{09:fig:opt_boxplot} shows this boxplot. Visually, the IQR
of D-SAX look to be the smallest, S-SAX the largest. The outlier spread is also
larger for S-SAX than for the other methods. The important variables regarding
\figref{09:fig:opt_boxplot} are summarized in \tabref{09:tab:opt-stats}.
\begin{figure}[H]
    \centering
    \includegraphics[width=\textwidth]{recall_boxplot}
    \caption{Boxplot showing the recall for the optimal parameter sets of each
    method. The full list of parameters can be found in \tabref{09:tab:opt-param}.}
    \label{09:fig:opt_boxplot}
\end{figure}
\begin{table}[H]
    \centering
    \caption{Table showing statistical measures for the recall of the optimal 
    parameter sets for each method. Best values are highlighted in bold.}
    \label{09:tab:opt-stats}
    \vspace{0.5em}
    \begin{tabular}{|>{\columncolor{tablegray}} c | r | r | r |}\hline
        \rowcolor{tablegray} 
            \diagbox{Method}{Measure} & IQR & Median & Outliers  \\\hline
        S-SAX   & 0.035 & \textbf{99.60\%} & 11               \\\hline
        D-SAX   & \textbf{0.030}    & 99.35\%    & \textbf{6}     \\\hline
        MSAX    & 0.033    & 99.13\%    & \textbf{6}     \\\hline
    \end{tabular}
\end{table}
\tabref{09:tab:opt-stats} shows that D-SAX has the best IQR, S-SAX has the
highest median recall, and both D-SAX and MSAX have the lowest number of
outliers. As a final test, a possible biserial correlation between the methods
and the resulting recall is investigated using R's standard \Verb_cor()_ 
function, which calculates the Pearson correlation. The correlation coefficient 
of the recall depending on S-SAX or D-SAX is 0.06, for the recall depending on
S-SAX or MSAX it is 0.033, and for the recall depending on D-SAX or MSAX it is
-0.04. All three correlation coefficients are very close to zero, which
indicates that there is no correlation between the methods used and the
resulting recall value.

\end{document}

