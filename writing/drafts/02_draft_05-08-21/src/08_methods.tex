\documentclass[../01_main.tex]{subfiles}

\begin{document}

\section{Methods}

\TODO{make a final applications section that explains how HOT SAX will be used
with each of the time series}

\TODO{why can filtering be ignored? put this as further research to investigate
the influence of filtering on this process}

\TODO{give good reasons why I chose the methods and data bases}

\TODO{goals:
\\- reader can assess believability of results
\\- all information necessary to replicate the research
\\- describe all materials, procedure, ect
\\- all the formulae
\\- state all the limitations of the methods and the ones I impose myself
\\- analytical methods and languages
}

\TODO{answer questions:
\\- can someone else accurately replicate the study
\\- can the data be obtained again
\\- are all parts / instruments described with enough accuracy
\\- is the data freely available
\\- can the statistical analysis be repeated
\\- can the algorithms be replicated?
}

\TODO{Sections:
    \\- general overview -> flowchart
    \\- then explain each element of the flowchart one by one
    \\- use formulae etc
    \\- nice amount of tikz graphs
    \\- section on implementation with details and the more important elements
    \\    use another flow chart?
    \\- use graphs to illustrate all important elements
    \\- make a data description section that describes my process of data
    handling; which database
    \\- explain the parameters that the methods have and what they mean
    \\- describe how a got all the data
}

This section explains the methods used in this research. \TODO{create flow
charts for all this shit to make it simpler}. First methods section for the
analytical methods in a mathematical way.

\subsection{Mathematical Foundations}

\begin{itemize}
    \item section for the workings
    \item explain theoretical foundations of the approach
    \item what is it grounded in, who, what, when
\end{itemize}

\subsection{SAX}

\begin{itemize}
    \item idea
    \item normalization
    \item dimensionality reduction
    \item discretization
    \item distance measure
    \item \TODO{all with graphs and formulas}
\end{itemize}

\subsection{MSAX}

\begin{itemize}
    \item idea
    \item normalization
    \item dimensionality reduction
    \item discretization
    \item distance measure
    \item \TODO{all with graphs and formulas}
    \item \TODO{points out differences to SAX}
\end{itemize}

\subsection{HOTSAX}

\begin{itemize}
    \item what is hotsax
    \item its theoretical foundations
    \item advantages, disadvantages
    \item how does it work
\end{itemize}

\subsection{Statistical Analysis of Results}

\begin{itemize}
    \item explain true positive, true negative, and so on
    \item explain recall, accuracy, precision, f1
    \item explain why recall was chosen and if that is fair
    \item introduce the correlations that we would expect to find if my
        hypothesis is true and also the ones that would disprove it
    \item which types of correlation, significance testing, and modeling will
        be used and why; what are the justifications
\end{itemize}

\subsection{Implementation}

How I implemented the above stuff. Languages, approaches, hurdles, all the
details needed to reproduce this research. Also mention the simplifications
I chose to make and why: no sliding window, only even divisors, only divisors
within sampling frequency and cutting ECG to even multiple of sampling
frequency.

\subsubsection{ECG acquisition}

flow chart for process

\begin{itemize}
    \item where to download
    \item what exactly are the ECGs
    \item where do they come from
    \item technical parameters of them
    \item the physionet suite
    \item annotations, what they mean, how I can get them, etc
\end{itemize}

\subsubsection{preprocessing}

flow chart for process

\TODO{the codes and constants given for each thing}

\begin{itemize}
    \item how were they preprocessed
    \item physionet suite
    \item my script and what it does and why
    \item problems and limitations of this 
    \item libraries used
\end{itemize}

\subsubsection{SAX}

\TODO{how was the whole data thing handled, how is the data created}

flow chart for process

\begin{itemize}
    \item how was sax implemented
    \item how does HOTSAX work here
    \item libraries used
\end{itemize}

\subsubsection{MSAX}

flow chart for process

\begin{itemize}
    \item how was sax implemented
    \item how does HOTSAX work here
    \item \TODO{point out differences to SAX}
    \item libraries used
\end{itemize}

\subsection{Statistical Evaluation}

\begin{itemize}
    \item reading the data into R
    \item summarizing the data
    \item the summarized data files
    \item libraries used
\end{itemize}

\end{document}
