\documentclass[../01_main.tex]{subfiles}

\begin{document}

 34 \subsection{Discussion}
 33
 32 \TODO{expand on this, potentially merge with results section}
 31
 30 In the previous section the results of the data analysis were presented. It was
 29 found that when the optimal parameters for MSAX and dual SAX are used, there is
 28 no significant difference between the two methods.
 27
 26 The most important result is the one mentioned above, as it indicates that, we
 25 infer, the MSAX representation is equal in performance to the dual SAX method
 24 when applied to ECGs using HOT SAX. This is congruent with the results of
 23 \mycite{anacleto2020}, where the authors found only a slight difference between
 22 the dual SAX method and MSAX. Discussing the actual parameters, MSAX used a PAA
 21 segment count of 6, representing 60 times reduction in dimension, while dual
 20 SAX performed best for $w=12$, a 30 times reduction in dimension.
 19
 18 \end{document}
 17 showed parameter
 16 sets with $k=-1$. The best parameter sets under these conditions had values of
 15 $w$ that were $\le 12$. Only the 7th to 10th best parameter combinations had $w
 14 \ge 12$. \TODO{make a table here}\TODO{do this for single and dual SAX}. As
 13 these values are average values, they have to investigated more closely with
 12 regards to their outliers and interquartile range. The following boxplot can
 11 illustrate the recall and precision rates for these parameter combinations.
 10
  9 \TODO{boxplots of each of the result sections}
  8 \TODO{top 10 table for each of the methods}
  7 \TODO{investigate correlation between sax and msax for those and see if it is
  6 significant}

\section{Discussion}

\TODO{expand on this, potentially merge with results section}

In the previous section the results of the data analysis were presented. It was
found that when the optimal parameters for MSAX and dual SAX are used, there is
no significant difference between the two methods. This shows, that for optimal
parameter settings, the MSAX representation is equal in performance to the 
dual SAX method when applied to ECGs using HOT SAX. This is congruent with 
the results of \mycite{anacleto2020}, where the authors found only a slight 
difference between the dual SAX method and MSAX. Discussing the actual 
parameters, MSAX used a PAA segment count of 6, representing 60 times reduction 
in dimension, while dual SAX performed best for $w=12$, a 30 times reduction in 
dimension.

\end{document}

\TODO{what do your results mean?
\\- make sure to distiguish results and interpretation "I infer that"...
\\- start with summary of important results
\\- major patterns
\\- relations, trends, generalizations
\\- exceptions
\\- likely causes for the things above?
\\- agreement / disagreement with previous work
\\- interpret with respect to hypothesis
\\- hypothesis testing here?
\\- other questions this relates to?
\\- consider all possibilities
\\- what new insight have we gained?
\\- include the supporting evidence for each line of reasoning
\\- what is the significance of the current results
\\- cite related results
}

\subsection{MSAX vs SAX}

\TODO{use the results from the previous section to come to a conclusion}

\begin{itemize}
    \item do proper hypothesis testing of my hypothesis statement
    \item argue which sets of parameters are the most effective
    \item judge if I proved what I set out to prove
    \item what are the uses of this method
    \item which applications could this fit?
    \item what should be done in future research
\end{itemize}

\end{document}
