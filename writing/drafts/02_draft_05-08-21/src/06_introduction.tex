\documentclass[../01_main.tex]{subfiles}

\begin{document}

\section*{Introduction}
\addcontentsline{toc}{section}{Introduction}

Heart diseases are the most deadly diseases on the planet, with 16\% of all
annual global deaths being caused by ischemic heart disease \mycite{who2020}. 
Ischemic heart disease is a condition caused by restricted blood flow to an
area of the heart which causes the heart muscle to not receive enough blood.
A restriction can be caused by a blood clot or by plaque buildup and if the
flow of blood is restricted too much, it can cause myocardial ischemia. 
Myocardial
ischemia, commonly known as a heart attack, is the result of oxygen-deprived
heart tissue dying \mycite{ihd2010}. Fortunately, ischemic heart disease can
be diagnosed before it causes a heart attack. One method of diagnosis involves
a stress test during which the heart's activity is recorded using an
electrocardiograph. The resulting recording reflecting the heart's activity is
called an electrocardiogram (ECG or EKG) \mycite{ihd2010}.\par

\begin{wrapfigure}{r}{0.5\textwidth}
    \centering
    \includegraphics[width=0.48\textwidth,height=0.4\textwidth]{06_ecg}
    \caption{An ECG wave.}
    \label{06:fig:ecg}
\end{wrapfigure}
The ECG is a diagnostic tool used to evaluate patients with (suspected) heart 
problems. It is a non-invasive, real-time, and cost-effective method
that may be used to diagnose IHD and other heart diseases like arrhythmia (the 
presence of irregular heartbeat rhythms). The ECG is the most common tool 
used for cardiac analysis and diagnosis \mycite{alghatrif2012,kligfield2007,
xie2020}. The electric measurements which an ECG records are taken in millivolts,
which represent the electrical activity of the heart with each heartbeat.
\figref{06:fig:ecg} shows the electrical activity an ECG records for a single
heartbeat. This is also called the ECG wave \mycite{kligfield2007,xie2020}.
When a disease affects the way the heart beats, an ECG can record those changes
and they can then be used to diagnose the disease.\par

Unfortunately, using an ECG to diagnose a cardiac condition is difficult. The
changes caused by diseases are often small and can be easily missed.
Furthermore, ECGs can be longer than 24 hours, and analyzing such a large amount
of data is very time-intensive \mycite{alghatrif2012,xie2020}. To mitigate this
issue, computers have been used to perform ECG analysis. Computers can be used 
to perform some or all of the steps involved in ECG analysis, which are listed 
below \mycite{kligfield2007}:
\begin{enumerate}
    \item signal acquisition and filtering,
    \item \label{06:enum:step2} data transformation or preparation for processing,
    \item waveform recognition,
    \item feature extraction, and
    \item \label{06:enum:step5} classification or diagnosis.
\end{enumerate}
Of the above steps, steps \ref{06:enum:step2} and \ref{06:enum:step5} are 
particularly interesting. Step \ref{06:enum:step2} is
important because, while computers can analyze data faster than humans, they,
too, have trouble analyzing large amounts of data \mycite{lin2003}. As part of
the data transformation step, the complexity and size of the ECG data can be
reduced. The methods used for this step are called time series representation
methods \mycite{aghabozorgi2015}. An established and widely used time series
representation is the Symbolic Aggregate Approximation (SAX). It is possible 
to apply time series representation methods to ECGs because ECGs are time 
series. A time series is a set of values recorded at specific 
times \mycite{brockwell2016} and ECGs are a special type of time series--a 
multivariate time series. Multivariate time series record multiple values for
each point in time \mycite{anacleto2020}, for ECGs these values are the
different leads. There are also time series representation methods for
multivariate time series. Multivariate SAX is an extension of SAX to
multivariate time series that incorporates the relationships between the
different elements, e.g. the ECG leads, and should make MSAX a better
representation than SAX for multivariate data \mycite{anacleto2020}.\par

ECG analysis step \ref{06:enum:step5} traditionally involves a cardiologist
manually looking at the heartbeats of an ECG to determine if they are normal or
if the heartbeat represents an anomaly \mycite{becker2006}. A type of time 
series anomaly are discords. Discords are segments of a time series that are 
very different from
the other segments of the time series. These discords can represent important
features of time series, e.g. possibly abnormal heartbeats in an ECG.
Accordingly, it is desirable to quickly find the discords in a time series.
This is called classification, each segment of the ECG is either classified as
a discord or as a non-discord.
While this can be done by comparing all segments of a time series to all other
segments, HOT SAX presents a more efficient solution. HOT SAX is a discord
discovery algorithm that uses the SAX representation and intelligent heuristics
to speed up the discord discovery process \mycite{keogh2005}. While HOT SAX
uses the SAX representation, the only requirements of the algorithm are that
two representations can be compared to each other. As MSAX is such
a representation \mycite{anacleto2020}, HOT SAX could be used with MSAX.\par

This work aims to investigate how the use of the MSAX representation
combined with the HOT SAX algorithm, called HOT MSAX hereinafter, will
influence ECG discord discovery performance compared to the standard HOT SAX
algorithm and which parameters achieve the best performance. The HOT SAX and 
HOT MSAX algorithms will be compared experimentally
using a well-known ECG database called the MIT-BIH Arrhythmia Database
\mycite{moody2001,goldberger2000}. The performance of the algorithms will be
primarily measured by their recall (also known as sensitivity). Recall is
relevant here because when it comes to discovering potentially medically
relevant discords, it is better to be a bit too diligent and mark too many
segments, lowering the accuracy, if it means that more discords can be
discovered. A robust and fast ECG discord discovery system could significantly 
speed up and simplify the 
work of a cardiologist by automatically classifying segments of an ECG into
discords and non-discords. The discords could then be analyzed first and, if
the method is sufficiently accurate, present the cardiologist with the
medically relevant segments very quickly. Classifying ECG anomalies with the HOT 
MSAX algorithm should achieve a greater recall value than with HOT SAX because,
unlike SAX, MSAX is designed to work with multivariate data like ECGs.\par

The following paper first contains a section reviewing the background and
work related to time series, ECGs, and their analysis. Secondly, this work's
methods, SAX, MSAX, and HOT SAX, are covered in detail. The third section
presents the results of this research. Fourth is a discussion of the results,
followed by a conclusion.

\end{document}
