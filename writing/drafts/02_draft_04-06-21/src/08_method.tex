\documentclass[../01_main.tex]{subfiles}

\begin{document}

\section{Methods}

this will lay out the methods used in this research and explain the previous
information in more mathematical detail

\subsection{What are time series}

time series are an ordered list of measurements generally taken at regular time
intervals. Insert mathematical definition here.

picture of a single lead of a time series

\subsection{Multivariate time series}

multivariate time series are time series that include more than one variable
for each point in time.

picture of ecg with graph of 2 concurrent leads.

\subsection{Where the ECGs come from}

explain the data bases; how the annotation get there; how can they be used? how
can a usable annotated version be extracted?

\subsection{Preprocessing of the ECGs}

Explain why normalization takes place, what the rationale is, and why it
matters. How is it done?

Distinguish between normalization for univariate and multivariate time series.

\subsection{Applying SAX and MSAX}

how does PAA work? what is the idea behind it? what are the advantages and
drawbacks?

How is the discretization done? 

Investigate the probability distribution for the SAX segments, is it close
enough to the expected normal gaussian distribution?

How are the alphabets created and how does the distance work?

example of a discretized time series, both uni- and multivariate

\subsection{Choice of parameters}

What are good parameters to choose for $w$, $a$, and the others? are there
algorithms, or is it a guessing game?

\subsection{Anomaly Detection or classification}

which method is being used for detection or classification? explain it
mathematically, how does it work for SAX/MSAX specifically?

\textbf{should I include implementation details here? or are those irrelevant?}

\subsection{how to test the effectiveness of these methods}

statistical conventions for testing

how can this be done in connection to the annotations in the files?

\end{document}
