\documentclass[../01_main.tex]{subfiles}

\begin{document}

\section{Discussion}

This section discussed the results presented in the previous section. It is
this paper's goal to investigate how HOT MSAX influences the ECG discord
discovery compared to HOT SAX. Additionally, the best parameters for
high-recall discord discovery were sought. During the experiment phase, two
datasets based on the MIT-BIH ECG database were used.\par

During the analysis of the first dataset, it was
found that both HOT SAX and HOT MSAX perform poorly when the subsequence length
parameter is not equal to the PAA segment count parameter. A possible cause of
this is the way annotations were counted as True Positives and True Negatives. 
By only counting
the annotations in the segment they fall into, a possible influence of the
discord on the following segments is neglected. This could lead to HOT SAX and
HOT MSAX identifying a discord caused by an abnormal heartbeat in the segment
following the annotation and thus lead to the detection being labeled
a False Positive. By making the subsequences as large as possible, the chance
of this happening becomes smaller and accordingly, the recall value would
increase. By setting the parameter $m$ equal to $w$ in the second dataset, the
number and proportion of parameter sets satisfying the 95\% recall condition
increased.\par

When the second dataset was being analyzed, every set of parameters in the top 
10 values of S-SAX, D-SAX, and MSAX had its
$k=-1$. The parameter $k$ controls how many of the discords identified by HOT
SAX and HOT MSAX will be returned after the computation is finished. Parameter
$k$ being equal to -1 for the optimal parameters regardless of the method
indicates that the highest concrete values of $k$, 300 for dataset 2, is too
low to enable top recall values. This behavior may be explained by considering
the precision values. If the average precision of HOT SAX and HOT MSAX is 
$\approx$ 36\% and the goal is to maximize recall, restricting the number of 
detected discords that are used will directly lower the recall value.\par

For all parameter sets in dataset 2, comparing S-SAX, D-SAX, and MSAX showed 
that MSAX performs acceptably for 5.1\%
of all parameter combinations in dataset 2. S-SAX performed acceptably for
1.2\% and D-SAX for 3.9\%. From this is it inferred that MSAX is more robust
when it comes to parameter choice. All three methods were tested with the same
parameters, but MSAX performed acceptably for a larger share than the other
methods. This is relevant because in applications involving unknown data, the
parameter selection cannot be optimized without testing. A more robust method
could still perform adequately in such a situation.\par

The optimal parameter sets (see \tabref{09:tab:opt-param}) for the three 
methods all had a value of $k=-1$,
which may be the result of the subsegment-annotation relationship mentioned
above. The PAA segment values and thus the subsequence lengths for D-SAX and
MSAX were 12 compared to 36 for S-SAX. The dimension reduction ratio for MSAX
and D-SAX is 30, while it is 10 for S-SAX. A possible explanation for this is
that MSAX and D-SAX are better suited to the multivariate discord detection 
tasks tested here. As a result of that, they do not require as much 
information as the
less-effective S-SAX method to achieve the same recall values. The information
requirement is relevant because, while PAA, SAX, and MSAX all lower-bound the
Euclidean Distance and thus accurately represent the raw time series, each
average trades a decrease in information about the shape of the time series for
a reduced dimension. If S-SAX is less effective at extracting information from
a multivariate time series, higher numbers of PAA segments may be required to
gain enough information to achieve high recall values.\par

The interquartile ranges of the recall value for all three optimal methods are not
significantly different, all being close to 0.03. The median recall value is very similar across
all three methods as well, roughly 99\%. The number of outliers is 11 for 
S-SAX and 6 for the
other two methods. The larger number of outliers for the S-SAX method could be
attributed to it not considering the multivariate nature of ECGs. All observed
outliers were outliers below the median recall value, indicating that certain
ECGs are more difficult for the method to analyze than others. This is
especially true for S-SAX, which does not properly take the nature of ECGs into
account. This result should not be surprising, as both D-SAX and MSAX use all
the information available, while S-SAX does not and is thus put at
a disadvantage.

A correlation between the optimal S-SAX, D-SAX, or MSAX methods
(\tabref{09:tab:opt-param}) and the recall 
value 
has not been found. As a result, the hypothesis of this paper cannot be
supported. There is no evidence of HOT MSAX improving the recall value compared
to HOT SAX. \textcite{anacleto2020}, in the paper introducing MSAX, analyzed
a different ECG dataset as part of a collection of different time series. They
used a $k$-Nearest-Neighbor classifier and found that there is no significant
difference between the SAX representation and the MSAX representation when
applied to ECG analysis. Their findings support this research's results.

\end{document}
