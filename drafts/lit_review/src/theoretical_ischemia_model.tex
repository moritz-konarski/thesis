\documentclass[../00_main.tex]{subfiles}

\begin{document}

\section{\textcite{l}}

\fullcite{l}

\subsubsection*{Summary}

\begin{itemize}
    \item 2D ventricular tissue model
    \item model Hyperkalemia (too much potassium), acidosis (lower
        conductivity)
    \item ischemia might leave traces that can be picked up using an ECG
    \item because ischemia is hard to properly study in the wild, having
        a model is very convenient
    \item they use the Luo--Rudy I cellular formulation for a 2D slice
    \item based on guinea--pig hearts
    \item equation of electrical activity of a stimulated cell
    \item the differential equations were solved using a combination of
        explicit (Euler) and implicit methods
    \item intracellular loss of potassium ions is the main alteration of
        electrical activity -- hyperkalemia
    \item electrical parameters were altered to simulate these conditions
    \item good description of why they tweaked which parameters for simulation
    \item myocardium cell model: 2D 
    \item cells are connected by gap junctions (low resistance "bridges")
    \item even a simple square--grid can be used to study complex phenomena
    \item simple impedance is enough to model the cell--to--cell interactions
    \item no--flux boundary conditions -- no current is leaving the system at
        the edges
    \item it needs to be sufficiently big to have some area that is not
        affected by the disturbance at the boundaries
    \item you excite a layer of 5 cells, then let the model run
\end{itemize}

\end{document}
