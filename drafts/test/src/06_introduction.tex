\documentclass[../01_main.tex]{subfiles}

\begin{document}

\section{Introduction}

\subsection{Important Points}

\subsubsection{background and purpose}

\begin{itemize}
    \item ischemia and similar diseases are some of the most deadly and common
        diseases
    \item when it comes to ischemic heart disease (IHD), rapid decision making
        is important
    \item ECG is one of the most widely used diagnostic tools
    \item reading an ECG is very difficult, which leads to different results
        among different physicians
    \item to develop software that analyzes 12--lead ECG to detect IHD
    \item this could reduce the time it takes to diagnose IHD, which is crucial
    \item detect changes during myocardial ischemia, some of those remain
        invisible to physicians
\end{itemize}

\subsubsection{goals}

\begin{itemize}
    \item create a 12--lead ECG analysis tool to diagnose IHD
    \item we will mathematically model the changes of the ECG compared to
        at--rest, nominal ECGs
\end{itemize}

\subsubsection{questions, problematic, rationale}

\begin{itemize}
    \item the ECG is the most widely used method to assess heart
        conditions
    \item the QRST--wave complex changes when ischemia is present, enabling its
        detection
    \item a mathematical model could make the analysis of ECGs easier for
        doctors and speed up their diagnosis
    \item the model needs to work well for this to be possible
    \item such a tool would remove some of the problems that normally exist
        (mentioned above)
\end{itemize}

\subsubsection{background, literature review}

\begin{itemize}
    \item heart disease is a significant medical issue
    \item one of the most deadly ones
    \item middle income countries like KG are hit harder
    \item health expenditure in KG is also one of the lowest
    \item IHD is the main killing disease
    \item for most treatment methods, the longer the treatment is delayed, the
        lower the chances of survival become
    \item if the necessary infrastructure is nonexistent, treatment times
        cannot be reduced to acceptable levels
    \item basically, in Kyrgyzstan most modern and good methods do not work
        because of the missing infrastructure and economic limits
    \item computers can help to analyze an ECG, which makes diagnosis easier
\end{itemize}

\subsubsection{methods}

\begin{itemize}
    \item get 100 digitized ECGs from healthy volunteers
    \item from this a good model of healthy and stressed ECGs should be created
    \item maybe use FFT for the analysis
    \item use a Maplesoft Signal Processing Tool for wave analysis
\end{itemize}

\subsection{Reference list}

\begin{enumerate}
    \item European Society of Cardiology: Cardiovascular Disease Statistics 
        2019. On behalf of the Atlas Writing Group European Heart Journal 
        (2020) 41, 12\_85.
    \item Gruntzig, A. Transluminal dilatation of coronary-artery stenosis. 
        Lancet 1978, 311, 1093.
    \item Sigwart, U.; Puel, J.; Mirkovitch, V.; Joffre, F.; Kappenberger, L. 
        Intravascular stents to prevent occlusion and restenosis after 
        transluminal angioplasty. N. Engl. J.Med. 1987, 316, 701–706.
    \item Stefanini,G.G.;Holmes,D.R.,Jr. Drug-eluting coronary artery stents. 
        N. Engl. J.Med. 2013, 368, 254–265.
    \item Pinto DS, Frederick PD, Chakrabarti AK, Kirtane AJ, Ullman E, Dejam 
        A, Miller DP, Henry TD, Gibson CM, National Registry of Myocardial 
        Infarction Investigators. Benefit of transferring ST-segment-elevation 
        myocardial infarction patients for percutaneous coronary intervention 
        compared with administration of onsite fibrinolytic declines as delays 
        increase. Circulation 2011;124(23):2512–2521.
    \item Armstrong PW, Gershlick AH, Goldstein P, Wilcox R, Danays T, Lambert 
        Y, Sulimov V, Rosell Ortiz F, Ostojic M, Welsh RC, Carvalho AC, Nanas 
        J, ArntzHR, Halvorsen S, Huber K, Grajek S, Fresco C, Bluhmki E, 
        Regelin A, Vandenberghe K, Bogaerts K, Van de Werf F, STREAM 
        Investigative Team. Fibrinolysis or primary PCI in ST-segment elevation 
        myocardial infarction. N Engl J Med 2013;368(15):1379–1387.
    \item Barbarash OL, Kashtalap VV. The place of pharmacoinvasive management
        in patients with ST-elevation acute coronary syndrome in Russia. 
        Kardiologiia. 2014;54(9):79-85. (In Russ).
    \item 2017 ESC Guidelines for the management of acute myocardial infarction 
        in patients presenting with ST-segment elevation. The Task Force for 
        the management of acute myocardial infarction in patients presenting 
        with ST-segment elevation of the European Society of Cardiology (ESC). 
        European Heart Journal (2018) 39, 119–177
    \item 2015 ESC Guidelines for the management of acute coronary syndromes in 
        patients presenting without persistent ST-segment elevation. European 
        Heart Journal (2016) 37, 267–315
    \item Navarese, E. P., Gurbel, P. A., Andreotti, F., Tantry, U., Jeong, Y. 
        H., Kozinski, M., et al. (2013). Optimal timing of coronary invasive 
        strategy in non-ST-segment elevation acute coronary syndromes: a 
        systematic review and meta-analysis. Ann. Intern. Med. 158, 261–270.
    \item Milasinovic, D., Milosevic, A., Marinkovic, J., Vukcevic, V., Ristic, 
        A., Asanin, M., et al. (2015). Timing of invasive strategy in NSTE-ACS 
        patients and effect on clinical outcomes: a systematic review and 
        metaanalysis of randomized controlled trials. Atherosclerosis 241, 
        48–54.
    \item Neumann FJ, Sousa-Uva M, Ahlsson A, Alfonso F, Banning AP, Benedetto 
        U, Byrne RA, Collet JP, Falk V, Head SJ, Juni P, Kastrati A, Koller A, 
        Kristensen SD, Niebauer J, Richter DJ, Seferovic PM, Sibbing D, 
        Stefanini GG, Windecker S, Yadav R, Zembala MO. 2018 ESC/EACTS 
        Guidelines on myocardial revascularization. Eur Heart J 2019;40:87\_165.
    \item A Validated Prediction Model for All Forms of Acute Coronary 
        Syndrome: Estimating the Risk of 6-Month Postdischarge Death in an 
        International Registry. Kim A. Eagle, MD; Michael J. Lim, MD; Omar H. 
        Dabbous, MD, MPH; et al. JAMA. 2004;291(22):2727-2733.
    \item TIMI Risk Score for ST-Elevation Myocardial Infarction: A Convenient, 
        Bedside, Clinical Score for Risk Assessment at Presentation. An 
        Intravenous tPA for Treatment of Infarcting Myocardium Early II Trial 
        Substudy. David A. Morrow, MD; Elliott M. Antman, MD; Andrew 
        Charlesworth, BSc; Richard Cairns, BSc; Sabina A. Murphy, MPH; James A. 
        de Lemos, MD; Robert P. Giugliano, MD, SM; Carolyn H. McCabe, BS; 
        Eugene Braunwald, MD. Circulation. 2000;102:2031-2037.
    \item 2019 ESC Guidelines for the diagnosis and management of chronic 
        coronary syndromes. The Task Force for the diagnosis and management of 
        chronic coronary syndromes of the European Society of Cardiology (ESC). 
        European Heart Journal (2020) 41, 407\_477
    \item Hakeem A, Ghosh B, Shah K, Agarwal S, Kasula S, Hacioglu Y, Bhatti S, 
        Ahmed Z, Uretsky B. Incremental prognostic value of post-intervention 
        Pd/Pa in patients undergoing ischemia-driven percutaneous coronary 
        intervention. JACC Cardiovasc Interv 2019;12:2002–2014.
    \item Jeremias A, Davies JE, Maehara A, Matsumura M, Schneider J, Tang K, 
        Talwar S, Marques K, Shammas NW, Gruberg L, Seto A, Samady H, Sharp A, 
        Ali ZA, Mintz G, Patel M, Stone GW. Blinded physiological assessment of 
        residual ischemia after successful angiographic percutaneous coronary 
        intervention: the DEFINE PCI study. JACC Cardiovasc Interv 
        2019;12:1991–2001.
    \item Myers, P.D., Scirica, B.M. \& Stultz, C.M. Machine Learning Improves 
        Risk Stratification After Acute Coronary Syndrome. Sci Rep 7, 12692 
        (2017). https://doi.org/10.1038/s41598-017-12951-x
    \item Goto S, Kimura M, Katsumata Y, Goto S, Kamatani T, Ichihara G, et al. 
        (2019) Artificial intelligence to predict needs for urgent 
        revascularization from 12-leads electrocardiography in emergency 
        patients. PLoS ONE 14(1): e0210103. https://doi.org/10.1371/
        journal.pone.0210103
    \item Kudenchuk PJ, Maynard C, Cobb LA, Wirkus M, Martin JS, Kennedy JW, 
        Weaver WD. Utility of the prehospital electrocardiogram in diagnosing 
        acute coronary syndromes: the Myocardial Infarction Triage and 
        Intervention (MITI) Project. Journal of the American College of 
        Cardiology.1998; 2(1):17–27.
    \item J.Z.Hemsey, K. Dracup, K. Fleischmann, C.E. Sommargren, and B.J.Drew. 
        Pre-hospital 12-Lead ST-Segment Monitoring Improves the Early Diagnosis 
        of Acute Coronary Syndrome. Electrocardiol. 2012 May ; 45(3): 266–271. 
        doi:10.1016/j.jelectrocard.2011.10.004
\end{enumerate}

\end{document}
