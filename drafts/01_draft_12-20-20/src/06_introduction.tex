\documentclass[../01_main.tex]{subfiles}

\begin{document}

\section{Introduction}

\subsection{\textcite{xie2020}}

\subsubsection{Introduction}

\begin{itemize}
    \item cvd is the leading cause of death worldwide
    \item 30\% of deaths and 130 million cases a year [1]
    \item ECG is good, non-invasive and real-time: heartbeat recognition, blood
        pressure detection, disease detection
    \item discovery of ECG [4]
    \item electronic analysis can give suggestions
    \item common ECG formats are 1-lead, 3-lead, 6-lead, 12-lead
    \item 12-lead is the standard and more detailed
    \item ECG is also future proof and becoming more readily available
    \item a doctor's reading of an ECG is heavily dependent on their
        experience, training, certs
    \item automatic analysis is becoming more and more common
    \item ECG features are unique information extracted that represent the
        state of the heart
    \item source [17] is a list of common feature classifiers
    \item instead of feature extraction and later classification, just using
        one neural network to do all the work is becoming more and more common
    \item ECD -- time-varying signal with small amplitude
    \item the signal needs to be significantly de-noised for approaches to work
    \item normally though, signals are disturbed by baseline drift, electrode
        contact noise, power-line interference
    \item severe baseline wandering can lead to misdiagnosis
    \item methods of denoising
        \begin{itemize}
            \item finding the QRS complex is usually hard because PLI and EMG
                mask it
            \item digital filtering, wavelet transform, empirical mode
                decomposition [25]
            \item digital filters are widely used for this, wavelet too
                [18,16,27]
            \item src [29] is a really good method apparently
            \item src [30] is also great
            \item src [32] is favorable
            \item src [33] is a different approach
            \item Butterworth filter
        \end{itemize}
    \item feature engineering
        \begin{itemize}
            \item Fourier transform for investigating a signal in the frequency
                domain
            \item FFT is useful and fast for feature extraction
            \item QRS is the most striking, can be used for heart rate
            \item FFT does not provide any information on the time of any of
                the components
            \item short-time FT gives time and frequency information -- we can
                either have good time and bad frequency or vice versa
            \item the wavelet transform has a time scale resolution scheme that
                makes this simpler
            \item wavelets are good for all frequencies because they are
                adaptive
            \item their high resolution can give them the edge
            \item there are many different options of wavelets that are good
                for different things
            \item src [71] is myocardial infarction
            \item DWT is a good computational tool to assess ECG changes
            \item for statistical and morphological features 
            \item higher-order statistics have proven to be good at ECG
                analysis
        \end{itemize}
    \item dimensionality reduction is important because while more feature mean
        more accuracy, they also increase the computational cost
    \item most data has correlated variables, meaning they can be ignored
    \item feature selection tries to select a subset of the original features
        and only select the best ones -- options are filters, wrappers, and
        embedded
    \item filters are the most simple version, they simply remove the redundant
        data and then return the relevant data
    \item filters use algorithms to assign scores to individual features
    \item filters are fast and independent of the classification, but they may
        not be super good or precise
    \item feature extraction reduces the dimension of the information but does
        not throw out information, which makes it more efficient and precise
    \item this includes primary component analysis and other types of analysis
    \item some features of an ECG appear randomly, also entropy, energy, and
        fractal dimension cannot be easily spotted with the naked eye
    \item kernels can be used for locally linear embedding
    \item some machine learning decision making algorithms are k nearest
        neighbors KNN, support vector machine SVM
    \item KNN is pretty simple and divides points into multiple group using
        distance; data imbalance is hard to overcome and they are expensive for
        high-dimensional data
    \item SVM has good training ability on small data sets and it is a good
        all-rounder
    \item there is no standard about the construction of a NN for ECG analysis
    \item a general end-to-end model seems to be the best solution, removing
        the need for optimization at each and every step -- feature extraction
        is shifted to the learning body, which is a nice solution
    \item a list of all the databases and what they are good at
    \item good list of applications of the whole thing
\end{itemize}

\subsection{Plan}

\begin{itemize}
    \item databases:
        \begin{itemize}
            \item MIT-BIH Normal Sinus Rhythm Database for normal ECGs
            \item European ST-T Database for ST and T wave changes -- patients
                with ischemia
            \item INCART database for ischemia, arrhythmias, coronary artery
                disease
            \item Lobachevsky University Electrocardiography Database for
                12-lead stuff for different cardiological diseases
            \item long therm ST database -- for st segment detection
            \item suggestions why only 5 minutes are used/necessary to detect
                stuff
        \end{itemize}
    \item use the Butterworth filter in the Julia DSP.jl package to filter the
        noise out
    \item use FFT, SFFT, Wavelet for feature extraction, also in julia if
        possible
    \item find some simple type of filter to do feature selection -- 
    \item classification could be done using the NearestNeighbors.jl package
\end{itemize}

\subsection{Outline}

\subsubsection{Problem Statement}

\begin{itemize}
    \item ischemia and similar diseases are some of the most deadly and common
        diseases
    \item IHD -- what is it? how can it be diagnosed (ECG)? how can it be
        treated(Stents)?
    \item what is the research problem that people are facing?
    \item the QRST–wave complex changes when ischemia is present, enabling its 
        detection
    \item heat disease is a significant and deadly medical issue
    \item poorer countries like Kyrgyzstan are disproportionately affected
        because many of the newer and better methods cannot be afforded
        / implemented
    \item health expenditure in KG is low, the lower it is the worse these
        conditions are 
    \item 
\end{itemize}

\subsubsection{Rationale -- Justification -- Why}

\begin{itemize}
    \item when it comes to ischemic heart disease (IHD), rapid decision making
        is important -- why
    \item ECG is one of the most widely used diagnostic tools -- why
    \item reading an ECG is very difficult, which leads to different results
        among different physicians -- relevance
    \item this could reduce the time it takes to diagnose IHD, which is
        crucial --
    \item detect changes during myocardial ischemia, some of those remain
        invisible to physicians
    \item promising method because other people are doing this
    \item what are the applications in practice?
    \item freely available ECGs on the internet -- MIT-BIH, European ST-T 
        database and the others
\end{itemize}

\subsubsection{Goals and Objectives}

\begin{itemize}
    \item to develop software that analyzes 12–lead ECG to detect IHD -- how
        will we do that?
    \item create a 12–lead ECG analysis tool to diagnose IHD
    \item mathematically model the changes in the ECG compared to at-rest and
        normal ECGs
    \item mathematical model and implementation that can speed up diagnosis
        (which is critical)
    \item get 100 digitized ECGs from healthy volunteers
    \item use FFT for analysis
    \item Fourier Transform, Fast Fourier Transform, Discrete Fourier Transform
    \item compare the different transforms for this specific problem
\end{itemize}

\section{Literature Review}

\subsection{Outline}

\subsubsection{Current State of the Problem}

\begin{itemize}
    \item advances in IHD treatment (see research proposal)
    \item current methods for ECG modeling
    \item what is the progress in using FFT and DFT to model ECGs
    \item 
\end{itemize}

\newpage

\subsection{Important Points}

\subsubsection{background and purpose}

\begin{itemize}
    \item ischemia and similar diseases are some of the most deadly and common
        diseases
    \item when it comes to ischemic heart disease (IHD), rapid decision making
        is important
    \item ECG is one of the most widely used diagnostic tools
    \item reading an ECG is very difficult, which leads to different results
        among different physicians
    \item to develop software that analyzes 12--lead ECG to detect IHD
    \item this could reduce the time it takes to diagnose IHD, which is crucial
    \item detect changes during myocardial ischemia, some of those remain
        invisible to physicians
\end{itemize}

\subsubsection{goals}

\begin{itemize}
    \item create a 12--lead ECG analysis tool to diagnose IHD
    \item we will mathematically model the changes of the ECG compared to
        at--rest, nominal ECGs
\end{itemize}

\subsubsection{questions, problematic, rationale}

\begin{itemize}
    \item the ECG is the most widely used method to assess heart
        conditions
    \item the QRST--wave complex changes when ischemia is present, enabling its
        detection
    \item a mathematical model could make the analysis of ECGs easier for
        doctors and speed up their diagnosis
    \item the model needs to work well for this to be possible
    \item such a tool would remove some of the problems that normally exist
        (mentioned above)
\end{itemize}

\subsubsection{background, literature review}

\begin{itemize}
    \item heart disease is a significant medical issue
    \item one of the most deadly ones
    \item middle income countries like KG are hit harder
    \item health expenditure in KG is also one of the lowest
    \item IHD is the main killing disease
    \item for most treatment methods, the longer the treatment is delayed, the
        lower the chances of survival become
    \item if the necessary infrastructure is nonexistent, treatment times
        cannot be reduced to acceptable levels
    \item basically, in Kyrgyzstan most modern and good methods do not work
        because of the missing infrastructure and economic limits
    \item computers can help to analyze an ECG, which makes diagnosis easier
\end{itemize}

\subsubsection{methods}

\begin{itemize}
    \item get 100 digitized ECGs from healthy volunteers
    \item from this a good model of healthy and stressed ECGs should be created
    \item maybe use FFT for the analysis
    \item use a Maplesoft Signal Processing Tool for wave analysis
\end{itemize}

\subsection{Advice from Imanaliev}

\begin{enumerate}
    \item Search for the recent advancements in published papers
    \item Search for the advancements in software of the related problems
    \item Study the Fourier Transform and Fast Fourier Transforms, and their 
        representation on chosen software
    \item Comparison of the different transforms for the related problem
    \item Scan of the paper based verified cardiograms and digitalising
    \item Comparison of the scanned graphs with the verified graphs
    \item Adjustment of the software parameters
    \item Error estimate
    \item Analysis of the results with doctors
    \item Real time method probation
    \item Adjustment of the parameters
    \item Thesis preparation and submission
    \item Scientific Paper preparation and submission
    \item Distribution of the results in media and analysis of references
    \item Adjustment of the parameters
\end{enumerate}

\subsection{Content requirements}

\subsubsection{Introduction}

\begin{itemize}
    \item short, verbal problem statement
    \item rational relevance of the selected topic
    \item formulates goals and objectives of the project
    \item refer to some information
    \item maybe a brief description of the main results
\end{itemize}

\subsubsection{Literature Review}

\begin{itemize}
    \item overview of the current state of the problem
    \item based on analysis of literary sources
    \item don't summarize sources, just give the important information they
        contain
    \item don't just call it "Literature Review", call it something like 
        "Mathematical models and methods of magnetotelluric monitoring"
\end{itemize}

\end{document}
